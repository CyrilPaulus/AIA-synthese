\chapter{Sélection de features}

Réduire le nombre de variable est utile car

\begin{itemize}
	\item on évite le sur-apprentissage (car on apprend sur des variables inutiles) et on améliore les performances du modèle
	\item on améliore l'interprétabilité
	\item les modèles sont plus rapides et moins coûteux
	\item on réduit tous les temps de calculs (si la sélection est elle-même rapide)
\end{itemize}

Il existe deux familles de méthodes :

\begin{itemize}
	\item la sélection de features : on cherche un petit (ou le plus petit) sous-ensemble de features qui maximisent la précision du modèle
	\item le ranking de features : on trie les variables selon leur importance à la prédiction de la sortie
\end{itemize}

A noter qu'on peut obtenir une sélection de feature avec un ranking : il suffit de sélectionner les $k$ premières features.

Trois approches existent dans la sélection de features :

\begin{itemize}
	\item filtre : sélection à-priori des variables, indépendamment d'un algorithme d'apprentissage supervisé
	\item embarqué (embedded) : sélection de features embarquée dans un algorithme d'apprentissage
	\item wrapper : utiliser la validation croisée pour trouver l'ensemble optimal de features pour un algorithme
\end{itemize}

\section{Techniques de filtre}

	L'idée est d'associer un score à chaque feature et de retirer celles dont le score est trop bas.
	
	Généralement, on utilise une fonction de score univariable, comme avec les arbres de décision ou avec des tests statistiques (t-test, chi$^2$, etc). Le nombre optimal de features peut être déterminé par de la validation croisée.
	
	Des approches de sélection multi-variable existent (dans les arbres de décision, etc), et peuvent être utiles ; des features pourraient être inutiles toutes seules (on a un petit score univariable), mais ensembles, elles peuvent parfaitement expliquer la classification.
	
	\dessin{95}
	
	Avantages et inconvénients :
	\begin{itemize}
		\item[+] l'approche univariable est rapide et scalable
		\item[+] on est indépendant de l'algorithme d'apprentissage supervisé
		\item[-] on ignore l'algorithme d'apprentissage, et donc potentiellement une bonne combinaison de variables
		\item[-] l'approche univariable ignore la dépendance entre les features
		\item[-] l'approche multivariable est beaucoup plus lente
	\end{itemize}
	
\section{Techniques embarquées}

Certains algorithmes d'apprentissage supervisé embarquent une sélection de features ; la recherche d'un sous-ensemble optimal de features se fait dans l'algorithme même.

Exemples :

\begin{itemize}
	\item le splitting des arbres de décision
	\item les méthodes d'ensemble à base d'arbres
	\item les valeurs absolues des poids dans une SVM linéaire : $\yh(\xu) = sgn(\sum_i w_i x_i + b)$
\end{itemize}

Avantages et inconvénients :
	
\begin{itemize}
	\item[+] généralement efficace en terme de temps de calcul
	\item[+] bonne intégration avec l'algorithme d'apprentissage
	\item[+] multivariable
	\item[-] spécifique à l'algorithme d'apprentissage
\end{itemize}

	\subsection{LASSO}
	
	Il s'agit d'un modèle linéaire avec une pénalisation L1, comme de la ridge regression ; le carré est remplacé par une valeur absolue.
	
	$$\min_{\beta} \sum_{i = 1}^N (y_i -(\beta_0 + \sum_j \beta_j x_j))^2 + \lambda \sum_j \vert \beta_j \vert$$
	
	Les courbes rouges représentent la valeur des premiers termes, sans régularisation.
	
	\dessin{96}
	
	
	Le problème est identique à
	
	$$\min_{\beta} \sum_i (y_i - \beta \xu_i)^2$$
	$$\text{tel que } \sum_j \vert \beta_j \vert < C \text{ (losange)}$$
	$$\text{ou  } \sum_j  \beta_j^2  < C \text{ (cercle)}$$
	
	Un losange est mieux car on a plus de chance d'arriver au sommet d'un losange qu'au sommet d'un cercle, donc d'avoir un $\beta_i$ nul. Pour $\vert \beta \vert^p$, si $p \leq 1$, on se dirige vers une sorte d'étoile à quatre branches (\ding{71}), ce qui augmente encore plus la probabilité.
	
	Si $\lambda$ augmente, les coefficients rétrécissent ; on obtient un classement des variables (celle qui dure le plus longtemps est la plus importante).
	
	Cette méthode est plus lente que la \textit{ridge regression} car il y a un problème d'optimisation, et on est limité aux méthodes linéaires.
	
\section{Techniques de wrapper}

On essaie de trouver le sous-ensemble de features qui maximise la qualité du modèle induite par l'algorithme d'apprentissage. Cette qualité est estimée par de la validation croisée. Vu que le nombre de sous-ensemble d'un ensemble de $p$ features est de $2^p$, tous les sous-ensembles ne sont pas évalués et des heuristiques sont nécessaires.

Plusieurs approches existent :

\begin{itemize}
	\item forward (ou backward) selection : on ajoute (retire) les variables qui diminuent le plus (augmentent le moins) l'erreur
	\item optimisation avec des algorithmes génétiques
\end{itemize}

Une approche populaire est l'élimination récursive de features. On suppose que l'on dispose d'un algorithme d'apprentissage qui peut ranker les features (ex : SVM, arbres de décision). A partir de l'ensemble complet de features, on itère :

\begin{itemize}
	\item on apprend un modèle à partir de l'ensemble courant de features
	\item on calcule le rang des features avec le modèle
	\item on retire la feature avec le plus petit rank
\end{itemize}

Au final, on garde l'ensemble de features qui donne la plus petite erreur par validation croisée.

\dessinS{97}{.45}

Avantage et inconvénients :

\begin{itemize}
	\item[+] l'ensemble de features est taillé sur mesure pour l'algorithme d'apprentissage
	\item[+] il est possible de trouver des interactions et de supprimer des variables redondantes
	\item[-] méthode susceptible au sur-apprentissage : il est parfois facile de trouver un petit sous-ensemble de variables bruitées qui donne de très bons résultats
	\item[-] coûteux, car il faut construire un modèle sur chaque sous-ensemble de variables
\end{itemize}


\section{Biais de sélection}

\paragraph{Mauvaise méthode de sélection}

A partir de la base de données, on sélectionne les $N$ meilleurs variables avec un filtre. On évalue ensuite un algorithme qui utilise ces $N$ variables par validation croisée sur la base de données.

Supposons une expérience artificielle où les variables sont choisies aléatoirement, de même que la classe de sortie. On effectue alors deux essais :

\begin{itemize}
	\item tree bagging sans sélection de feature : erreur en validation croisée 10-fold = 52\%
	\item idem mais avec les 20 meilleures features : 35\%
\end{itemize}

On pourrait supposer qu'il y a 20 variables intéressantes et qu'avec elles on pourrait créer un meilleur classificateur qu'un classificateur aléatoire. Cependant, sur un nouvel ensemble d'échantillons, on obtient une erreur de 52\%. De plus, le 35\% obtenu n'est pas possible, car toutes les valeurs sont aléatoires : tout modèle aura une erreur de 50\%.

On a un problème de sur-apprentissage, car on a sélectionné et testé le modèle sur base de toute la base de données. Le bon protocole aurait dû être

\begin{itemize}
	\item diviser $\LS$ en 10 sous-ensembles
	\item pour $i = 1$ jusqu'à 10 :
	
	\begin{itemize}
		\item retirer le $i$ème sous-ensemble de $\LS$
		\item sélectionner les 20 variables sur les sous-ensembles restant
		\item apprendre le modèle sur les 20 variables et les objets restant
		\item tester le modèle sur le $i$ème sous-ensemble de variables
	\end{itemize}
\end{itemize}

Il faut donc utiliser une table différente pour la sélection et la validation, afin d'éviter de la corrélation lors de la validation.