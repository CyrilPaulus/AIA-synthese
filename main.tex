\documentclass[10pt,a4paper]{report}
\usepackage[utf8]{inputenc}
\usepackage{amsmath}
\usepackage{amsfonts}
\usepackage[french]{babel}
\usepackage{amssymb}

\usepackage[cm]{fullpage}

\usepackage{listings}
\usepackage{color}
\usepackage{verbatim}
\usepackage{framed}
\usepackage{ulem}
\usepackage{pigpen}

\usepackage{graphicx}
\newcommand{\ens}[1]{\lbrace #1 \rbrace}
\newcommand{\abs}[1]{\vert #1 \vert}

\newcommand{\union}{\cup}
\newcommand{\intersection}{\cap}
\newcommand{\mand}{\wedge}
\newcommand{\mor}{\vee}

\newcommand{\bigoh}{\mathcal{O}}

\newcommand{\dessin}[1]{\begin{center}\includegraphics[scale=0.6]{images/#1.png}\end{center}}
\newcommand{\dessinS}[2]{\begin{center}\includegraphics[scale=#2]{images/#1.png}\end{center}}

\newtheorem{theorem}{Théorème}[section]
\newtheorem{lemma}[theorem]{Lemma}
\newtheorem{proposition}[theorem]{Proposition}
\newtheorem{corollary}[theorem]{Corollary}

\newcommand{\qed}{\nobreak \ifvmode \relax \else
      \ifdim\lastskip<1.5em \hskip-\lastskip
      \hskip1.5em plus0em minus0.5em \fi \nobreak
      \vrule height0.75em width0.5em depth0.25em\fi}


\newenvironment{proof}[1][Preuve]{\begin{trivlist}
\item[\hskip \labelsep {\bfseries #1}]}{\qed\end{trivlist}}
\newenvironment{definition}[1][Définition]{\begin{trivlist}
\item[\hskip \labelsep {\bfseries #1}]}{\end{trivlist}}
\newenvironment{example}[1][Exemple]{\begin{trivlist}
\item[\hskip \labelsep {\bfseries #1}]}{\end{trivlist}}
\newenvironment{remark}[1][Remark]{\begin{trivlist}
\item[\hskip \labelsep {\bfseries #1}]}{\end{trivlist}}


%% Raccourcis, histoire de ne pas devenir fou avec les notations
\newcommand{\ey}[1]{E_y\{#1\}}
\newcommand{\els}[1]{E_{LS}\{#1\}}
\newcommand{\yh}{\hat{y}}				% y hat
\newcommand{\xh}{\hat{x}}				% x hat
\newcommand{\vary}[1]{var_y\{#1\}}
\newcommand{\varyx}[1]{var_{y\vert \underline{x}}\{#1\}}
\newcommand{\varls}[1]{var_{LS}\{#1\}}

\newcommand{\pyo}{P(\overline{y})}		% P(y barre)
\newcommand{\yo}{\overline{y}}			% y overline
\newcommand{\ao}{\overline{a}}			% a overline
\newcommand{\xo}{\overline{x}}			% x overline

\newcommand{\ab}{\textbf{a}} % a bold
\newcommand{\cb}{\textbf{c}} % c bold
\newcommand{\wb}{\textbf{w}} % w bold
\newcommand{\yb}{\textbf{y}} % w bold

\newcommand{\exy}{\text{E}_{\underline{x}, y}}
\newcommand{\eyx}[1]{\text{E}_{y \vert \underline{x}} \ens{#1}}
\newcommand{\ex}{\text{E}_{\underline{x}}}
\newcommand{\xu}{\underline{x}}			% x underline
\newcommand{\uu}{\underline{u}}			% u underline

\newcommand{\yens}{y_{\text{ens}}}
\newcommand{\yhens}{\hat{y}_{\text{ens}}}

\newcommand{\LS}{\text{LS}}
\newcommand{\TS}{\text{TS}}
\newcommand{\VS}{\text{VS}}
\newcommand{\GS}{\text{GS}}

\newcommand{\remarque}[1]{\textcolor{red}{#1}} % Note sur la synthèse

\title{Synthèse Apprentissage inductif appliqué}
\author{Jean-Philippe Collette}

\makeindex
\begin{document}	
	\maketitle
	\tableofcontents
	\chapter{Introduction}

L'apprentissage consiste à

\begin{itemize}
	\item améliorer les performances d'un ordinateur dans certaines tâches, avec de l'expérience ;
	\item extraire un modèle d'un système en se basant sur les observations de ce systèmes dans certaines situations ;
	\item créer un modèle, c'est-à-dire une relation entre les variables utilisées pour décrire le système.
\end{itemize}

Les deux buts principaux de l'apprentissage sont la prédiction et la meilleure compréhension d'un système.

L'apprentissage est utilise quand il n'y a pas d'expertise humaine, quand les humains ne sont pas capables d'expliquer leur expertise, quand les solutions changent au cours du temps ou quand les solutions nécessitent d'être adaptées à des cas particuliers.

\dessin{1}

L'exploration de données se déroulent en plusieurs étapes :

\begin{enumerate}
	\item Génération de données ;
	\item Préprocessing : normalisation des valeurs, traitement des valeurs manquantes, sélections d'une composante, etc ;
	\item Apprentissage : développement d'une hypothèse, choix d'un algorithme d'apprentissage, etc ;
	\item Validation d'hypothèse : validation croisée, déploiement du modèle, etc.
\end{enumerate}

	\section{Glossaire}
	
	\dessin{2}


	
	\part{Apprentissage supervisé}
	\chapter{Introduction}

\section{Comparaison des méthodes}
	
\dessin{23}

A noter que l'importance relative des critères dépend de l'application, et que ce ne sont que des tendances générales.

	\chapter{Arbres de décision}

Il s'agit d'un algorithme d'apprentissage qui peut gérer les problèmes de classification (binaire ou avec plusieurs valeurs) et avec des attributs qui peuvent être discrets ou continus.

Un arbre de décision est un arbre où

\begin{itemize}
	\item chaque noeud intérieur teste un attribut,
	\item chaque branche correspond à la valeur d'un attribut, et
	\item chaque feuille est labellisée par une classe.
\end{itemize}

\dessin{16}

L'apprentissage avec cet algorithme consiste à choisir la structure d'arbre et à déterminer les prédictions aux feuilles.

Afin de minimiser l'erreur de classification, on associe aux feuilles la classe majoritaire lorsqu'on descend dans l'arbre.

\section{Arbre de classification}
	\subsection{Création d'un arbre de classification}
	
	On a l'algorithme suivant (Top-down induction of DT), pour une procédure \textit{learn\_dt($\LS$)}, où $\LS$ est l'échantillon d'apprentissage.
	
	\begin{itemize}
		\item[$\bullet$] si tous les objets de LS ont la même classe, créer une feuille avec cette classe comme label,
		\item[$\bullet$] sinon,
		\begin{itemize}
			\item trouver le meilleur attribut $A$ pour une séparation,
			\item créer un noeud de test pour cet attribut, et
			\item pour chaque valeur $a$ de $A$,
			\begin{itemize}
				\item[$\circ$] construire $LS_a = \ens{o \in LS \vert A(o) = a}$, et
				\item[$\circ$] utiliser \textit{learn\_dt($LS_a$)}, pour créer un sous-arbre à partir de $LS_a$.
			\end{itemize}
		\end{itemize}
	\end{itemize}
	
	Propriétés : 
	
	\begin{itemize}
		\item sous-optimal mais rapide
		\item dépend très fortement du critère pour sélectionner l'attribut
	\end{itemize}
	
	\subsection{Recherche du meilleur attribut}
	
	Pour trouver le meilleur attribut, il faut définir un score afin d'évaluer les séparations possibles. Ce score devra favoriser la séparation en classe, afin de réduire la profondeur de l'arbre.
	
	\dessin{17}
	
		\subsubsection{Impureté}
		
		Une mesure assez commune est l'impureté. Soit $p_j$ la proportion d'objets de la classe $j$ ($j = 1, \dots , J$) dans $\LS$.
		
		\begin{itemize}
			\item $I(\LS)$ est minimal si $p_i = 1$ et $p_j = 0$ pour $j \neq i$.
			\item $I(\LS)$ est maximal si $p_j = \frac{1}{J}$ : il y a le même nombre d'objets de chaque classe.
		\end{itemize}
		
		Le meilleur split est celui qui maximise la réduction d'impureté.
		
		$$\Delta I(\LS, A) = I(\LS) - \sum_a \frac{\vert \LS_a \vert}{\vert \LS \vert} I(\LS_a)$$
		
		avec $\LS_a$ le sous-ensemble des objets de $\LS$ tels que $A = a$. $\Delta I$ est une mesure de score ou un critère de séparation.
		
		%$$I(LS, A) = H(LS) - \frac{\vert LS_{\text{left}}\vert}{\vert LS\vert}H(LS_{\text{left}}) - \frac{\vert LS_{\text{right}}\vert}{\vert LS \vert}H(LS_{\text{right}})$$
		
		
		Exemple de mesure d'impureté :
		
		\begin{itemize}
			\item entropie de Shannon : $H(\LS) = \sum_j p_j \log p_j$. Elle mesure l'incertitude, la surprise : une réduction de l'entropie est un gain d'information.
			
			\item indice de Gini : $I(\LS) = \sum_j p_j(1 - p_j)$
			\item taux d'erreur de malclassification : $I(\LS) = 1 - \max_j p_j$
			
		\end{itemize}
		
		\dessin{88}
		
	
	\subsection{Élagage}

	Le sur-apprentissage survient lorsqu'on considère un arbre trop profond et qu'on tient compte de bruits ou de mauvaises mesures dans les données.
	
	\dessin{18}
	
	Pour éviter l'overfitting, on a trois façons :
	
	\begin{itemize}
		\item pre-pruning/pré-élagage : arrêter d'étendre l'arbre plus tôt, avant qu'il n'atteigne le point où il classe parfaitement l'échantillon d'apprentissage ;
		\item post-pruning/post-élagage : permettre à l'arbre de sur-apprendre et de l'élaguer ensuite ;
		\item les méthodes Ensemble.
	\end{itemize}
	
	
	
		\subsubsection{Pre-pruning}
		
		On arrête la séparation de noeud si
		
		\begin{itemize}
			\item le nombre d'objets est trop petit (\textit{min\_samples\_split}), ou si
			\item l'impureté est trop faible, ou si
			\item le meilleur test n'est pas statistiquement pertinent
		\end{itemize}
		
		Le problème est que le paramètre optimal pour cet élagage est très dépendant du problème, et que l'on peut passer à côté de l'arbre optimal.
		
		
		\subsubsection{Post-pruning}
		
		On sépare l'ensemble d'apprentissage $\LS$ en deux :
		
		\begin{itemize}
			\item un ensemble $\GS$ (growing sample) pour construire l'arbre, et
			\item un ensemble de validation $\VS$ pour évaluer l'erreur de généralisation.
		\end{itemize}
		
		On va construire un arbre complet à partir de $\GS$, puis on va calculer un ensemble d'arbres $\ens{T_1, T_2, \dots}$, avec $T_1$ l'arbre complet et $T_i$ l'arbre obtenu en retirant des noeuds de l'arbre $T_{i - 1}$.
		
		On sélectionne ensuite le $T_i*$ de la séquence qui minimise l'erreur sur $\VS$.
		
		\dessin{19}
		
		Pour construire la séquence, on a plusieurs manières :
		
		\begin{itemize}
			\item reduced error pruning : à chaque étape, on retire le noeud qui diminue le plus l'erreur sur $\VS$
			\item cost-complexity pruning : on définit un critère de coût-complexité :
			
			$$\text{Error}_{\GS}(T) + \alpha \text{ Complexity}(T)$$
			
			On construit la séquence d'arbres qui minimise ce critère en augmentant $\alpha$.
		\end{itemize}
		
		
		Le problème du post-pruning est qu'il faut dédier une partie de l'ensemble d'apprentissage à la validation, ce qui peut poser un problème pour des petites bases de données. La solution est d'utiliser de la validation croisée $k$-fold.
	
	\subsection{Variables numériques}
	
	Deux solutions :
	\begin{itemize}
		\item pré-discrétiser, assigner des valeurs symboliques à des ranges (par exemple "froid" si la température est inférieure à 70$\,^{\circ}$F, "normal" si entre 70 et 75$\,^{\circ}$F, "chaud" si plus de 75$\,^{\circ}$F) ;
		\item discrétiser durant l'opération de construction de l'arbre.
		
		\dessin{20}
	\end{itemize}
	
	Slides 38 $\rightarrow$ 44 : variables numériques, attributs à plusieurs valeurs, valeurs d'attribut manquantes.
	
\section{Arbre de régression}
	
	Un arbre de régression est un arbre de décision où les labels des noeuds sont numériques.
	

	\dessin{89}
	
	Afin de minimiser l'erreur quadratique sur $\LS$, la prédiction à une feuille est la moyenne de tous les éléments de $\LS$ qui l'atteignent.
	
	L'impureté d'un échantillon est définie par la variance de la sortie de cet échantillon :
	
	$$I(\LS) = \text{var}_{y \vert \LS} \ens{y} = \text{E}_{y \vert \LS}((y - \text{E}_{y \vert \LS}(y))^2)$$
	
	Le meilleur split est celui qui diminue le plus la variance :
	
	$$\Delta I(\LS, A) =\text{var}_{y \vert \LS} \ens{y} - \sum_a \frac{\vert \LS_a \vert}{\vert \LS \vert} \text{var}_{y \vert \LS_a} \ens{y}$$

	L'élagage fonctionne de la même façon que pour les arbres de classification, si ce n'est que dans le post-pruning, c'est l'arbre qui minimise l'erreur quadratique sur $\VS$ qui est sélectionné.
	
	En pratique, le pruning est plus important en régression car les arbres complets sont beaucoup plus complexes ; souvent, tous les objets ont une sortie différence, du coup l'arbre complet a autant de feuilles qu'il y a d'objets dans l'échantillon d'apprentissage.


\section{Interprétabilité et sélection d'attribut}
	
	Un arbre de décision est très interprétable, il peut être converti facilement en un ensemble de règles "si \dots alors".
	
	Si certains attributs ne sont pas nécessaire pour la classification, il n'apparaitront pas dans l'arbre (élagé/pruned). C'est important si la mesure de certaines variables est coûteuse.
	
	Les arbres de décision sont souvent utilisés comme pre-processing pour d'autres algorithmes d'apprentissage, qui souffrent de variables inutiles.
	
	Certaines variables ont une importance, elles ne contribuent pas toutes de manière égale. Grâce aux arbres, on peut évaluer leur importance. Cela peut permettre notamment d'éviter de mesurer des attributs si cela est coûteux et inutile.
	
	L'importance d'une variable est calculée par la réduction de l'impureté :
	
	$$I(A) = \sum_{\substack{\text{noeuds où} \\ \text{A est testé}}} \vert \LS_{text{noeud}} \vert \Delta I(\LS_{\text{noeud}}, A)$$
	
	
	\section{Avantages et inconvénients}
	
	\begin{itemize}
		\item[+] très rapide et scalabe, on peut traiter d'énorme quantité d'entrées et d'objets ; la complexité est de l'ordre $\bigoh (nN\log N)$
		\item[+] donne une bonne interprétabilité et quantifie l'importance des variables ;
		\item[+] très flexible : support de plusieurs types d'attribut, de valeurs manquantes, de problèmes de classification et de régression, etc
		\item[-] grande variance et donc grande instabilité ;
		\item[-] souvent pas aussi précis que d'autres méthodes.
	\end{itemize}
	
	\subsection{$k$-NN - méthode des $k$ème plus proche voisin}
		
Cette méthode consiste à prédire la sortie en se basant sur les plus proches voisins de l'entrée.
		
\dessin{11}
		
Pour ce faire, on trouve les $k$ plus proches voisins, en utilisant la distance euclidienne. La sortie sera,
		
\begin{itemize}
	\item dans le cadre d'une classification, la classe la plus fréquente,
	\item dans le cadre d'une régression, la valeur moyenne.
\end{itemize}
		
\dessin{12}
		
\begin{itemize}
	\item[+] très simple ;
	\item[+] peut être adapté pour tout type de données, en changeant la mesure de la distance ;
	\item[-] choisir une bonne mesure de la distance est un problème compliqué ;
	\item[-] cet algorithme est très sensible à la présence de bruit ;
	\item[-] lent.
\end{itemize}	
	\chapter{Méthodes linéaires}

	\section{Introduction}
	Le but est de trouver un modèle qui est combinaison linéaire des entrées.

	\begin{itemize}
		\item Pour une régression, $y = w_0 + w_1 x_1 + w_2 x_2 + \dots + w_n x_n$
		\item Pour une classification, $y = c_1$ si $w_0 + w_1 x_1 + w_2 x_2 + \dots + w_n x_n > 0$, $c_2$ sinon.
	\end{itemize}

	\dessin{13}

	Plusieurs méthodes existent pour trouver les coefficients $w_i$ : 

	\begin{itemize}
		\item Régression : least-square regression, ridge regression, partial least square, support vector regression, LASSO, \dots
		\item Classification : linear discriminant analysis, PLS-discriminant analysis, support vector machines, \dots \\
	\end{itemize} 

	\begin{itemize}
		\item[+] simple ;
		\item[+] il existe des variantes rapides et scalable ;
		\item[+] cette méthode offre des modèles interprétatifs, à travers des poids variables (magnitude et signe) ;
		\item[-] parfois pas aussi précis que des autres méthodes (non linéaires).
	\end{itemize}

		\subsection{Ridge regression}
	
		On doit trouver $w$ qui minimise (avec $\lambda > 0$) :
	
		$$\sum_i (y_i - wx_i)^2 + \lambda \vert \vert w \vert \vert^2$$
	
		Dans le cadre d'un algèbre simple, si $X$ est la matrice d'entrée et $y$ le vecteur de sortie, la solution est donnée par
	
		$$w^r = (X^T X + \lambda I)^{-1}X^Ty$$
	
		$\lambda$ permet de réguler la complexité, et d'éviter des problèmes liés à la singularité de $X^TX$.
	
		\subsection{Perceptron}
	
		On doit trouver $w$ qui minimise
	
		$$\sum_i (y_i - wx_i)^2$$
	
		en utilisant une descente de radiant : étant donné un exemple d'entrainement $(x, y)$,
	
		$$\delta \leftarrow y - w^Tx$$
		$$\forall jw_i \leftarrow w_j + \eta \delta x_j$$
	
		Il s'agit d'un algorithme de type \textit{online}, c'est-à-dire qu'il traite les exemples un à un, alors qu'un algorithme de type \textit{batch} traite tous les exemples en une seule fois.
	
		La complexité est régulée par le taux d'apprentissage $\eta$ et le nombre d'itérations.
	
		\subsection{Extensions non linéaires}
	
		Plusieurs extensions existent :
	
		\begin{itemize}
			\item la généralisation des méthodes linéaires :
		
			$$y = w_0 + w_1 \phi_1(x_1) + w_2 \phi_2(x_2) + \dots + w_n \phi_n(x_n)$$
		
			N'importe quelle méthode linéaire peut être appliquée, mais la régulation devient plus importante.
		
			\item Réseaux de neurones artificiels, avec une seule couche cachée : si $g$ est une fonction non linéaire (par exemple une sigmoid),
		
			$$y = g(\sum_j w_J g(\sum_i w_{i, j} x_i))$$
		
			C'est une fonction non linéaire d'une combinaison linéaire de fonction non linéaires de combinaisons linéaires d'entrées.
		
			\item Méthode à base de noyaux :
		
			$$y = \sum_i w_i \phi_i(x) \Leftrightarrow y = \sum_j \alpha_j k(x_j, x)$$
		
			où $k(x, x') = \langle \phi(x), \phi(x') \rangle$, le produit scalaire dans l'espace donné et où $j$ indexe les exemples d'entraînement du modèle.
		\end{itemize}

	\section{Régression linéaire}
	
	Une régression linéaire essaie d'approximer la sortie avec
	
	$$\yh(o) = w_0 + \sum_{i = 1}^n w_i a_i(o)$$
	
	Ce modèle doit être linéaire dans les paramètres, pas nécessairement dans les entrées originales. On a donc, si $\phi$ est une transformation,
	
	$$\yh(o) = w_0 + \sum_{i = 1}^n w_i \phi_i(\textbf{a}(o))$$
	
	Les entrées peuvent être
	
	\begin{itemize}
		\item des mesures,
		\item des transformations de mesures (log, racine carrée, etc),
		\item des extensions, pas exemple $a_2(p) = a^2_1(o)$, et
		\item des valeurs "dummies".
	\end{itemize}
	
	\section{Solution minimisant l'erreur quadratique moyenne}
	
	Si on pose $a_0(o) = 1 \: \forall o$, et si on dénote
	
	
	\begin{itemize}
		\item $\ab'(o_i) = (a_0(o_i), a_1(o_i), \dots , a_n(o_i))^T$, et
		\item $\wb' = (w_0, w_1, \dots , w_n)^T$,
	\end{itemize}
	
	on a le carré de l'erreur (Square Error) sur un échantillon $i$ :
	
	$$SE(o_i, \wb') = (y(o_i) - \yh(o_i))^2 = (y(o_i) - \wb'^T\ab'(o_i))^2$$
	
	On a aussi le carré de l'erreur sur tous les objets de $LS$ (Total Square Error), avec $A' = (\ab'^1, \dots , \ab'^N)$
	
	$$TSE(LS, \wb') = \sum_{i = 1}^N (y(o_i) - \wb'^T\ab'(o_i))^2 = (\yb - A'^T\wb)^T(\yb - A'^T\wb')$$
	
	$$A' = \begin{array}{c}\left. \underbrace{\begin{pmatrix}
	1 & 1 & \dots & 1 \\ 
	a_1(o_1) & a_2(o_1) & \dots & a_N(o_1) \\ 
	\vdots & \dots & \dots & \vdots \\ 
	a_1(o_n) & a_2(o_n) & \dots & a_N(o_n)
	\end{pmatrix}}_{N}\right\} n + 1\end{array} $$
	
		\subsection{SE à une dimension}
		
		Avec une seule entrée, la solution est donnée par
		
		$$(w_0*, w_1*) = arg \min_{w_0, w_1} \sum_{i = 1}^N (y(o_i) - w_0 - w_1a_1(o_i))^2$$
		
		Lorsqu'on annule la dérivée, on obtient
		
		$$w_1* = \frac{cov(a_1, y)}{\sigma^2_{a_1}}$$
		$$w_0* = \yo - w_1*\overline{a}_1$$
		
		avec $\overline{a}_1 = \frac{1}{N} \sum_{k = 1}^Na_1(o_k)$ et $\yb = \frac{1}{N} \sum_{k = 1}^N y(o_k)$. Ainsi le seul $w_i$ à trouver est $w_0$. La meilleur valeur est la moyenne des $\yh(o_i)$.
		
		$$\yh = w_0, \: w_0* = arg \min_{w_0} \underbrace{\sum_{i = 1}^N (w_0 - y(o_i))^2}_{TSE(\LS, w_0)}$$
		
		$$\frac{d}{dw_0} TSE = \sum_{i = 1}^N 2(w_0 - y(o_i)) = 0$$
		$$\Leftrightarrow 2 N w_0 = 2 \sum_{i = 1}^N y(o_i) \Leftrightarrow w_0 = \frac{1}{N} \sum_{i = 1}^N y(o_i)$$
		
		Si on substitue dans l'équation $y(o) = w_0* + w_1*a_1(o)$, on a
		
		$$\frac{y(o) - \yo}{\sigma_y} = \rho_{a_1, y} \frac{a_1(o) - \overline{a}_1}{\sigma_{a_1}}$$
		
		$\rho_{a_1, y}$ est le coefficient de corrélation entre $a_1$ et $y$, et $\sigma_y$ et $\sigma_{a_1}$ sont les déviations standards de $y$ et $a_1$.
		
		\subsection{SE à plusieurs dimensions}

	\subsection{Réseaux de neurones artificiels}
		
55 $\rightarrow$ 60
	\chapter{Biais et variance}

Dans le cadre de l'apprentissage supervisé, on cherche une fonction $f : \mathcal{X} \rightarrow \mathcal{Y}$ qui minimise l'erreur de généralisation, soit l'espérance

$$E_{x, y} \{L(y, f(x))\}$$

La fonction $L : \mathcal{Y} \times \mathcal{Y} \rightarrow \mathbb{R}$ mesure la distance entre ses arguments :

\begin{itemize}
	\item en classification, $L(y, y') = 1$ si $y \neq y'$ (taux d'erreur)
	\item en régression, $L(y, y') = (y - y')^2$ (erreur quadratique).
\end{itemize}

Soit la fonction $\hat{f}_{LS}$ apprise à partir de $\LS$ à partir d'un algorithme d'apprentissage. Cette fonction (qui donne la prédiction à un certain point) est une variable aléatoire.

\dessin{24}

On peut définir l'erreur de généralisation, qui est utile pour caractériser et sélectionner un modèle :

$$Err_{LS} = E_{x, y} \{L(y, \hat{f}_{LS}(x))\}$$

Pour un algorithme d'apprentissage donné, l'erreur de généralisation sur des ensembles $\LS$ aléatoires de taille $N$ permet de caractériser un algorithme et est donnée par

$$E_{LS}\{Err_{LS}\} = E_{LS}\{E_{x, y} \{L(y, \hat{f}_{LS}(x))\} \}$$

	\section{Problème de régression sans variable d'entrée}
	
	Supposons que l'on cherche à prédire le mieux possible la taille d'un mâle adulte belge. On choisit une mesure de l'erreur, comme l'erreur quadratique, et on cherche une estimation $\hat{y}$ tel que l'espérance $E_y\{(y - \hat{y})^2\}$ sur toute la population belge soit minimale.
	
	\dessin{25}
	
	L'estimation qui minimise l'erreur peut être calculée en prenant 
	
	\begin{eqnarray*}
	 & \frac{\phi}{\phi y'} \ey{(y - y')^2} & = 0 \\
	\Leftrightarrow & \ey{-2(y - y')} & = 0 \\
	\Leftrightarrow & \ey{y} - \ey{y'} & = 0 \\
	\Leftrightarrow & y' & = \ey{y}
	\end{eqnarray*}
	
	Dans un modèle de Bayes, l'estimation qui minimise l'erreur est $\ey{y}$, soit la moyenne sur toute la population.
	
	En pratique, ce genre de valeur n'est pas calculable (il faudrait mesurer tous les adultes mâles belges).
	
	Vu que $p(y)$ est inconnu, on va trouver une estimation $\yh$ à partir d'un ensemble $LS = \ens{y_1, y_2, \dots , y_N}$, composé d'éléments tirés de manière aléatoire de la population. Autrement dit, on va chercher un algorithme d'apprentissage, par exemple
	
	\begin{enumerate}
		\item la moyenne : $\yh_1 = \frac{1}{N} \sum_{i = 1}^N y_i$
		\item en pondérant la valeur qu'on attend, avec $\lambda$ : $\yh_2 = \frac{\lambda . 180 + \sum_{i = 1}^N y_i}{\lambda + N}$, avec $\lambda \in [ 0, +\infty [ $
	\end{enumerate}
	
	\section{Décomposition biais/variance}
	
	Vu que $LS$ est tiré aléatoirement de la population, la prédiction $\yh$ est aussi une variable aléatoire. La distribution dépend de l'ensemble de l'apprentissage.
	
	\dessinS{26}{.5}
	
	Un bon algorithme d'apprentissage doit être bon sur un seul $LS$, mais aussi en moyenne sur plusieurs échantillons d'apprentissage. On veut minimiser 
	
	$$E = \els{\ey{(y - \yh)^2}}$$
	
	Si on ajoute et retire $\ey{y}$ et si on distribue le carré, on a
	
	\begin{eqnarray*}
  	E & = & \els{\ey{(y \textcolor{red}{- \ey{y} + \ey{y}} - \yh)^2}} \\
  	 & = & \els{\ey{(y - \ey{y})^2}} \\
  	 & + & \els{\ey{(\ey{y} - \yh)^2}} \\
  	 & + & \els{\ey{2(y - \ey{y})(\ey{y} - \yh)}}
	\end{eqnarray*}

	Or,
	
	\begin{itemize}
		\item $\els{\ey{(y - \ey{y})^2}} = \ey{(y - \ey{y})^2}$, car aucun terme ne dépend de $\LS$ (il n'y a que $\yh$ qui en dépend) ;
		\item $\els{\ey{(\ey{y} - \yh)^2}} = \els{(\ey{y} - \yh)^2}$ car $\ey{y}$ et $\yh$ sont des constantes dans l'expression, on peut donc les sortir du $\ey{.}$ qui les englobe.
	\end{itemize}
	
	\begin{eqnarray*}
  	E & = & \ey{(y - \ey{y})^2} \\
  	 & + & \els{(\ey{y} - \yh)^2} \\
  	 & + & \els{2\ey{(y - \ey{y})(\ey{y} - \yh)}}
	\end{eqnarray*}
	
	Or, $\ey{y - \ey{y}} = \ey{y} - \underbrace{\ey{\ey{y}}}_{\ey{y}}$, le troisième terme s'annule donc, ce qui conduit à
	
	$$E =\ey{(y - \ey{y})^2} + \els{(\ey{y} - \yh)^2}$$
	
	$\ey{(y - \ey{y})^2} = \vary{y}$ est la variance des données, c'est l'erreur résiduelle ; il s'agit de la plus petite erreur que l'on peut atteindre. Il s'agit d'une donnée du problème, on ne peut pas agir dessus.
	
	$\els{(\ey{y} - \yh)^2}$ est la moyenne sur tous les ensembles d'apprentissage de l'erreur du modèle de Bayes.
	
	%\dessinS{27}{.5}
	
	Si on suit le même raisonnement
	
	\begin{eqnarray*}
  	\els{(\ey{y} - \yh)^2} & = & \els{(\ey{y} \textcolor{red}{- \els{\yh} + \els{\yh}} - \yh)^2} \\
  	 & = & \els{(\ey{y} - \els{\yh})^2} \\
  	 & + & \els{(\els{\yh} - \yh)^2} \\
  	 & + & \els{2(\ey{y} - \els{\yh})(\els{\yh} - \els{\yh})} \\
  	 & = & (\ey{y} - \els{\yh})^2 \\
  	 & + & \els{(\yh - \els{\yh})^2} \\
  	 & + & 2(\ey{y} - \els{\yh})(\els{\yh} - \els{\yh}) \\
  	 & = & (\ey{y} - \els{\yh})^2 + \els{(\yh - \els{\yh})^2}
	\end{eqnarray*}
	

	%\dessinS{28}{.5}
		
	$\els{\yh}$ est la moyenne du modèle sur tout l'échantillon $LS$. $(\ey{y} - \els{\yh})^2$ est le biais$^2$, l'erreur entre le modèle de Bayes et le modèle moyen.
		
	%\dessinS{29}{.5}
	
	$\varls{\yh}$ est l'estimation de la variance, qui est la conséquence de sur-apprentissage. On a que
	
	\begin{eqnarray*}
	E & = & \vary{y} + \text{biais}^2 + \els{(\yh - \els{\yh})^2} \\
     & = & \vary{y} + \text{biais}^2 + \varls{\yh}
	\end{eqnarray*}
	\dessinS{30}{.5}
	
	
	
	Pour les exemples :
	
	\begin{itemize}
		\item $\yh_1 = \frac{1}{N} \sum_{i = 1}^N y_i$
		
		$$\els{\frac{1}{N} \sum_{i = 1}^N y_i} = \frac{1}{N} \sum_{i = 1}^N E_{LS}\{y_i\} = \frac{1}{N} \sum_{i = 1}^N \ey{y} = \ey{y}$$
		
		$\els{y_i}$ est la taille moyenne d'une même personne pour plusieurs $LS$, tandis que $\ey{y}$ est la taille moyenne de la population : ce sont les mêmes choses, car la distribution est iid, le choix ne fait pas varier la taille.
		
		$$\varls{\yh} = \varls{\frac{1}{N} \sum_{i = 1}^N y_i} = \frac{1}{N^2} \varls{\sum_{i = 1}^N y_i}$$
		
		Par une propriété.
		
		$$= \frac{1}{N^2} \sum_{i = 1}^N \varls{y_i}$$
		
		car les $y_i$ sont indépendants.
		
		$$= \frac{1}{N^2} N \vary{y}$$
		
		car la distribution est iid.
		
		Les caractéristiques de cet estimateur sont que
		
		\begin{enumerate}
			\item si $N \rightarrow + \infty$, alors $E \rightarrow 0$.
			\item le biais est nul, c'est le meilleur estimateur.			
		\end{enumerate}
		
		\item $\lambda$ : $\yh_2 = \frac{\lambda . 180 + \sum_{i = 1}^N y_i}{\lambda + N}$, avec $\lambda \in [ 0, +\infty [ $
		
		$$\text{biais}^2 = \frac{\lambda}{\lambda + N} (\ey{y} - 180)^2$$
		
		$$\varls{\yh_2} = \frac{N}{(\lambda + N)^2} \vary{y}$$
		
		Caractéristiques :
		
		\begin{itemize}
			\item le biais est petit si la taille moyenne d'un homme belge est proche de 180
			\item plus l'échantillon est grand, plus le biais est petit
			\item si $\lambda$ est élevé, la variance va diminuer (car la prédiction devient constante), mais le biais va augmenter.
		\end{itemize}
	\end{itemize}
	
	$\yh_1$ et $\yh_2$ sont des estimateurs consistants car lorsque $N$ augmente (donc lorsqu'on a plus d'échantillons), $E$ diminue, ce qui n'est pas toujours le cas. $\yh_2$ combat le sur-apprentissage, on dit que $\lambda$ permet la régulation.
	
	
		\subsection{Approche bayesienne}
		
		Supposons que
		
		\begin{itemize}
			\item la taille moyenne est proche de 180 cm, suivant une loi normale :
			
			$$\pyo = A.exp(- \frac{(\yo - 180)^2}{2 \sigma_{\yo}})$$
			
			\item la taille d'une personne est gaussienne par rapport à la moyenne :
			
			$$P(y_i \vert \yo ) = B.exp(- \frac{(y_i - \yo)}{2 \sigma_y})$$
		\end{itemize}
		
		Quelle est la valeur la plus probable $\yo$ en connaissant $LS$ ?
		
		$$\yh = arg \: \text{max}_{\yo} P(\yo \vert LS)$$
		
		Voir slide 20.
	
		\subsection{Problème de régression (complet)}
	
		On considère qu'on cherche une fonction $\yh(x)$ à plusieurs entrées, on a une moyenne sur tout l'espace d'entrée.
	
		L'erreur devient
	
		\begin{eqnarray*}
		E & = & \els{\exy{(y - \yh(\xu))^2}} = \ex{\els{\eyx{(y - \yh(\xu))^2}}} \\
		 & = & \ex{\varyx{y}} + \ex{\text{biais}^2(\xu)} + \ex{\varls{\yh(\xu)}}
		\end{eqnarray*}
	
		On a finalement l'erreur suivante :
	
		$$\els{\eyx{(y - \yh(\xu))^2}} = \text{noise}(\xu) + \text{biais}^2(\xu) + \text{variance}(\xu)$$
	
		avec
	
		\begin{itemize}
			\item $\text{noise}(\xu) = \eyx{(y - h_B(\xu))^2}$, qui quantifie de combien $y$ varie de $h_B(\xu) = \eyx{y}$, le modèle de Bayes
			\item $\text{biais}^2(\xu) = (h_B(\xu) - \els{\yh(\xu)})^2$, qui mesure l'erreur entre le modèle de Bayes et la moyenne du modèle
			\item $\text{variance}(\xu) = \els{(\yh - \els{\yh(\xu)})^2}$, qui quantifie de combien $\yh(\xu)$ varie d'un échantillon d'apprentissage à un autre.
		\end{itemize}
	
		\subsection{Illustration}
		
		Soit une fonction $y = h(x) + \epsilon$, $\epsilon \approx N(0, 1)$.
		
		\dessinS{105}{.3}
		
		Si on utilise une méthode avec trop de biais et peu de variance, on a du sous-apprentissage. Le biais est la différence au carré entre les courbes bleu et noire (intégrée).
		
		\dessinS{106}{.3}
		
		Si on utilise une méthode avec peu de biais et beaucoup de variance, on a du sur-apprentissage. Le décalage est faible, mais la variance est haute. La variance est proche du bruit.
		
		\dessinS{107}{.3}
		
		A noter que ne pas avoir de bruit ne signifie pas qu'il n'y a pas de variance, mais qu'il y en a moins.
		
		\dessinS{108}{.3}

		\subsection{Lien entre biais/variance et sous/sur-apprentissage}
	
		Un modèle trop simple ne permet pas de capturer toute la complexité des données : on a un biais élevé.
	
		Un modèle trop complexe conduit à du sur-apprentissage : on a une variance élevée.
	
	
	\section{Problèmes de classification}
	
	L'erreur moyenne de mauvaise classification est
	
	$$E = \els{E_{\underline{x}, y}\{1 (y \neq \yh(\xu))\}}$$
	
	Le meilleur modèle possible est le modèle de Bayes :
	
	$$h_B(\xu) = \text{arg } \max_c P(y = c \vert \xu)$$
	
	Le modèle moyen est
	
	$$\text{arg } \max_c P(\yh(\xu) = c \vert \xu)$$
	
	Il n'y a pas de décomposition de l'erreur de mauvaise classification moyenne en un biais et une variance, mais on observe les mêmes phénomènes.
	
	Le biais est à peu près une composant de l'erreur qui est systématique et indépendant de l'échantillon d'apprentissage. La variance est l'erreur due à la variabilité du modèle par rapport à l'aspect aléatoire de l'échantillon d'apprentissage.
	
	\section{Paramètres influençant le biais et la variance}
	
		\subsection{Complexité du modèle}
		% Slide 31 ?
		
		Pour le problème considéré, on a que la variance du bruit vaut 1 et la moyenne est nulle : l'erreur résiduelle vaut donc 1.
		
		\dessinS{31}{.45}
		
		Généralement, le biais est une fonction décroissante de la complexité, tandis que la variance croît.
		
			\subsubsection{Réseaux de neurones}
		
			\dessinS{32}{.45}
			
			Plus on augmente le nombre de neurones dans la couche cachée et plus on obtient une fonction complexe. Du coup, on tient forcément plus compte de bruit ou d'outliers, ce qui entraîne une augmentation de la variance dans les prédictions. Le biais va diminuer pour la même raison ; on approxime de mieux en mieux la fonction.
		
			\subsubsection{Arbres de régression}
			
			\dessinS{33}{.45}
			
			Le raisonnement est similaire à celui des réseaux de neurones : plus l'arbre est profond et plus on tient compte du bruit et de cas isolés dans les données. De ce fait, on aura une erreur de resubstitution (donc le biais) moindre, mais la variance dans les prédictions sera plus grande.
			
			\subsubsection{$k$-NN}
			
			\dessinS{34}{.45}
			
			Plus $k$ est grand et plus on considère des voisins éloignés. Il est logique que la variance diminue, car on considère une fraction de plus en plus grande des données dans le calcul de la prédiction, du coup les prédictions varient de moins en moins (si on considérait tous les points, la prédiction serait la même tout le temps). En revanche, vu qu'on considère des voisins qui ont peu de chance d'être similaires à l'entrée, on aura un plus grand biais.
		
		\subsection{Bruit}
		
		A une complexité de modèle fixée, le biais augmente avec la complexité du modèle de Bayes. Cependant, les effets sur la variance sont difficiles à prédire.
		
		A cause du bruit, la variance augmente, mais globalement le biais n'est pas affecté. Par exemple, avec un arbre de régression (complet).
		
		\dessinS{35}{.45}
		
		\subsection{Taille de l'échantillon d'apprentissage}
		
		Pour les modèles dont la complexité ne dépend pas de la taille de $\LS$, le biais reste constant et la variance diminue avec la taille de l'échantillon. Par exemple, avec une régression linéaire.
		
		\dessinS{36}{.45}
		
		Lorsque la complexité du modèle est dépendante de la taille de l'échantillon d'apprentissage (par exemple les arbres), le biais et la variance décroissent avec la taille de l'échantillon d'apprentissage. Par exemple, avec un arbre de régression.
		
		\dessinS{37}{.45}
		
		\subsection{Algorithmes d'apprentissage}
		
		\dessinS{38}{.5}
		
		
		\begin{itemize}
			\item la régression linéaire possède peu de paramètres, donc une petite variance. Par contre, si la classification est non linéaire, on a une variance élevée
			\item $k$-NN : un petit $k$ implique une variance élevée et un biais modéré. Un grand $k$ implique une petite variance mais un biais plus grand
			\item réseau de neurones : le biais est petit, mais la variance augmente avec la complexité du modèle
			\item arbre de régression : le biais est petit, mais la variance est grande.
		\end{itemize}
	\section{Techniques de réduction du biais et de la variance}
	
	Afin de réduire le biais et la variance, on peut jouer sur les paramètres de l'algorithme d'apprentissage, cependant cela ne suffit pas toujours. On peut alors passer aux méthodes d'ensemble, qui offrent un trade-off biais/variance différent, mais qui font perdre des caractéristiques de la méthode initiale.
	
		\subsection{Réduction de la variance}
		
		L'idée générale est de réduire la capacité de l'algorithme d'apprendre $\LS$. Cela se fait par
		
		\begin{itemize}
			\item du pruning, afin de réduire explicitement la complexité de l'algorithme
			\item un arrêt précoce, afin de réduire la recherche
			\item une régularisation, qui réduit l'espace d'hypothèses. On utilise par exemple le weight decay avec les réseaux de neurones, qui consiste à pénaliser les grandes valeurs pour les poids.
		\end{itemize}
		
		\dessinS{98}{.45}
		
		La réalisation est la vraie valeur auquel s'ajoute du bruit. Le biais du modèle n'est pas équivalent au biais de l'algorithme d'apprentissage.
				
		Exemple de résultats :
		
		\dessinS{99}{.45}
		
		Comme attendu, la variance diminue mais le biais augmente.
		
		A noter qu'une bonne estimation du biais est l'erreur de resubstitution (sur $\LS$).
	\chapter{Évaluation de modèles}

Pour un modèle donné et ayant appris sur un ensemble de données (de taille $N$), on aimerait estimer ses performances. Le but est de

\begin{itemize}
	\item sélectionner un ou plusieurs modèles (ex : déterminer la bonne complexité ou choisir entre différents algorithmes d'apprentissage).
	\item évaluer les modèles, afin d'estimer les performances sur des nouvelles données.
\end{itemize}


\section{Erreurs}

On peut s'arranger pour qu'un modèle minimise l'erreur de re-substitution, ou l'erreur de généralisation.

L'erreur de re-substitution ($\LS$ error) est l'erreur obtenue en appliquant le modèle à l'échantillon d'apprentissage. Plus le modèle est complexe et plus cette erreur sera proche de 0.

L'erreur de généralisation est l'erreur obtenue sur la prédiction de nouvelles données. Si $\fh_{\LS}$ est la fonction apprise sur un échantillon $\LS$, l'erreur de généralisation est décrite par

$$\exy\ens{L(y, \fh_{\LS}(x))}$$

avec $L(., .)$ une fonction de perte qui mesure la différence entre les arguments.

Cette erreur est à différencier de l'erreur de généralisation attendue (expected generalization error) sur un $\LS$ aléatoire de taille $N$, décrite par $E_{\LS} \ens{\text{Err}_{\LS}} = \els{\exy{L(y, \fh_{\LS}(x))}}$.

\section{Méthodes d'évaluation}
	\subsection{Méthode test set}

	On suppose qu'on dispose de beaucoup de données, que $N$ est grand. On divise la base de données en deux parties, une qui servira d'ensemble d'apprentissage et l'autre d'ensemble de test (par ex 70\%, 30\%). La méthode est la suivante :

	\begin{itemize}
		\item on apprend le modèle sur $\LS$
		\item on le test sur $\TS$
		\item l'estimation qui en résulte est une estimation de l'erreur d'un modèle qui aurait appris sur toute la base de données.
	\end{itemize}
	
	Avantages et inconvénients :
	
	\begin{itemize}
		\item[+] très simple à mettre en place ;
		\item[+] efficace ;
		\item[-] les données d'apprentissage sont moindres ;
		\item[-] instable lorsque la base de données est petite.
	\end{itemize}
	
	\dessin{65}
	
	\subsection{K-fold}
	
	La méthode du test-set n'est pas fiable sur une petite base de données car elle est basée sur un petit échantillon d'une base de données déjà réduite. De plus, on utilise l'estimation du modèle construit comme une estimation pour toute la base de données, or lorsque la base de données est très petite, apprendre le modèle sur toute la base ou sur une partie change fortement les résultats.
	
	On peut tracer la courbe d'apprentissage, où on confronte les performances à la taille de l'échantillon d'apprentissage. On voit qu'il faut que le $\LS$ soit d'une taille minimale si on veut avoir des résultats pertinents.
	
	\dessin{63}
	
	On utilise pour cela la validation k-fold : on divise aléatoirement la base de données en $k$ sous-ensembles (typiquement $k = 10$).
	
	\dessin{64}
	
	La méthode est la suivante :
	
	\begin{itemize}
		\item pour chaque sous-ensemble
		\begin{itemize}
			\item apprendre le modèle sur les objets qui ne sont pas dans le sous-ensemble
			\item calculer les prédictions du modèle sur les points du sous-ensemble
		\end{itemize}
		\item reporter l'erreur moyenne sur ces prédictions
	\end{itemize}
	
	Lorsque $k = N$, la méthode s'appelle une validation croisée \textit{leave-one-out}.
	
	Le choix de $k$ est très important et conduit à différents avantages et inconvénients :
	
	\begin{itemize}
		\item si $k = N$ (leave-one out) :
		\begin{itemize}
			\item[+] non-biaisé : enlever un objet ne change pas trop la taille de l'échantillon d'apprentissage
			\item[-] grande variance : on dépend énormément de la base de données
			\item[-] lent : il faut entraîner $N$ modèles
		\end{itemize}
		\item si $k = 5, 10$ :
		
		\begin{itemize}
			\item[+] petite variance et rapide : on n'a que $5-10$ modèles sur peu de données
			\item[-] potentiellement biaisé (voir courbe d'apprentissage)
		\end{itemize}
	\end{itemize}
	
	
\subsection{Impact de la complexité d'un modèle avec une validation croisée}

	\dessin{7}

	Il vaut mieux utiliser une petite complexité si on ne dispose pas de beaucoup d'échantillons.

	Avec une complexité fixe, on obtient le comportement suivant.

	\dessin{8}

	Le contrôle de la complexité s'appelle la régulation ou le lissage (smoothing). Il peut être contrôlé de plusieurs façons :

	\begin{itemize}
		\item en variant la taille de l'espace d'hypothèse, autrement dit le nombre de modèles candidats, la valeur des paramètres, etc ;
		\item avec un critère de performance : on oppose les performances de l'ensemble d'apprentissage et la valeur des paramètres, autrement dit minimiser
	
		$$\text{Err}(LS) + \lambda C(\text{model})$$
	
		\item avec des algorithmes d'optimisation : le nombre d'itération, la nature du problème d'optimisation, etc.
	\end{itemize}

	Le choix d'un algorithme se fait en comparant leur taux d'erreur avec une méthode de type cross-validation sur des sous-échantillons. Ensuite, l'algorithme avec le plus petit taux est utilisé comme modèle prédicatif sur toutes les données.

	L'utilisation intensive de la méthode CV peut entraîner du sur-apprentissage. En effet, plus on compare de modèles complexes, plus on a une chance d'en trouver un qui convient pour les données. La solution pour éviter cela est de réserver un ensemble de test additionnel (ou de les générer), et de l'utiliser pour tester les performances du modèle final.
	
	\subsection{Bootstrap}
	
	Un échantillon bootstrap est un échantillon avec un remplacement ; des objets n'apparaissent pas et d'autres apparaissent plusieurs fois.
	
	\dessin{66}
	
	On a alors que
	
	$$P(o_i \in \text{ bootstrap}) = 1 - (1 - \frac{1}{N})^N \approx 1 - \frac{1}{e} = 0.632$$
	
	L'idée est d'utiliser les 30\% de données qu'il reste comme $\TS$. On peut alors estimer l'erreur de bootstrap :
	
	\begin{itemize}
		\item pour $i = 1$ jusqu'à $B$ :
		
		\begin{itemize}
			\item prendre un échantillon de boostrap $B_i$ de la base de données
			\item apprendre un modèle $f_i$ sur cet échantillon
		\end{itemize}
		
		\item pour chaque objet, calculer l'erreur de tous les modèles qui ont été construis sans lui (environ 30\%)
		\item moyenner sur tous les objets
	\end{itemize}
	
	Des améliorations existent :
	
	\begin{itemize}
		\item $.632$ bootstrap : correction pour la courbe d'apprentissage
		\item $.632+$ bootstrap : correction pour le sur-apprentissage
	\end{itemize}
	
	\subsection{Erreurs de test conditionnelles et erreurs de test attendues}
	
	Pour un modèle $\hat{f}_\LS$ donné, on a l'erreur de test conditionnelle :
	
	$$\text{Err}_\LS = \exy{L(y, \hat{f}_\LS(x))}$$
	
	On a l'erreur attendue :
	
	$$\els{\text{Err}_\LS} = \els{\exy{L(y, \hat{f}_\LS(x))}}$$
	
	Seule la méthode test set estime la première erreur, la validation croisée permet quant à elle d'estimer la seconde car on fait de l'apprentissage sur des $\LS$ différents.
	
\section{Méthodes de sélection}

Le but, pour une base de données de $N$ objets, est de déterminer le meilleur modèle possible et d'estimer l'erreur des prédictions. La méthodologie dépend encore une fois de la taille de la base.

Le choix d'une mesure de l'erreur ou de la qualité dépend fortement de l'application. On peut également définir des autres critères pour évaluer un modèle.

	\subsection{Méthode test set}
	
	Si la base de données est grande, on divise aléatoirement l'ensemble d'apprentissage en trois parties : un ensemble d'apprentissage $\LS$, un ensemble de validation $\VS$ et un ensemble de test $\TS$ (par exemple 50\%, 25\% et 25\%).
	
	\dessin{67}
	
	La méthode est la suivante :
	
	\begin{itemize}
		\item apprendre les modèles à comparer sur $\LS$
		\item sélectionner le meilleur en basant les performances sur $\VS$
		\item le ré-entraîner sur $\LS + \VS$
		\item le tester sur $\TS$, afin d'avoir une estimation des performances
		\item le ré-entraîner sur $\LS + \VS + \TS$, afin d'avoir le modèle final
	\end{itemize}
	
	$\VS$ permet de sélectionner le modèle. Une fois que c'est fait, on réapprend sur $\LS$ et $\VS$ et on teste sur $\TS$. Le choix de la meilleure méthode (apprise sur $\LS$ et testée sur $\VS$) n'entraîne pas de sur-apprentissage, mais il y a tout de même un risque s'il y a trop de méthodes à tester (on en aurait une qui donne des résultats par coup de chance), ce qui pourrait donner une grosse erreur sur $\TS$. La solution est d'ajouter une nouvelle boucle de validation croisée ; tout choix d'échantillon peut créer un biais.
	
	\subsection{Validation croisée}
	
	On utilise deux étapes de validation croisée en k-fold.
	
	\dessin{68}
	
	CV1 est utilisé pour évaluer le modèle final, tandis que CV2 est utilisé pour la sélection du modèle.
	
	On peut également combiner la méthode test set et la validation croisée.
	
	\dessin{69}
	
	Les deux étapes sont nécessaires, car plus on compare des modèles et plus la probabilité de tomber par chance sur celui qui donne des bons résultats augmente. Les erreurs sur $\VS$ ou sur CV2 sont alors en général trop optimistes.
	
	Illustration :
	
	\dessinS{70}{.4}
	
	\subsection{Méthodes analytiques}
	
	On cherche le modèle qui minimise un critère de la forme
	
	$$\text{Err}(\LS) + G(\text{complexité})$$
	
	où $G$ est une fonction monotone croissante. Le critère est dérivé de preuves théoriques.
	
	L'avantage est qu'il n'y a pas de re-entraînement, par contre on ne peut l'utiliser que pour la sélection de modèle, et on pourrait manquer le vrai optimum dans le cas d'échantillons finis.
	
\section{Biais de sélection}
	
	En général, n'importe quel choix fait en utilisant la sortie doit être dans une boucle de validation croisée, car il peut potentiellement amener à du sur-apprentissage. En effet, un gain d'apprentissage sur un $\LS$ peut ne pas se répercuter sur des nouvelles données à prédire.
	
	\dessinS{71}{.4}
		
\section{Mesure de performance}

	\subsection{Classification binaire}
	
	Les résultats peuvent être résumés dans un tableau de contingence/matrice de confusion.
	
	\dessinS{72}{.5}
	
	On définit alors
	
	$$\text{Taux d'erreur (error rate) } = \frac{FP + FN}{N + P}$$
	$$\text{Précision (accuracy) } = \frac{TP + TN}{N + P} = 1 - \text{ error rate}$$
	
	Le taux d'erreur est cependant limité : on n'a pas d'informations sur la distribution des erreurs sur les classes, et il est sensible aux changement dans la distribution des classes dans l'échantillon de test.
	
	\dessinS{73}{.45}
	
	Dans le cadre de diagnostiques médicaux, on utilise des mesures plus appropriées :
	
	$$\text{Sensibilité (sensitivity/recall) } = \frac{TP}{P}$$
	$$\text{Spécificité (specificity) } = \frac{TN}{TN + FP} = 1 - \frac{FP}{N}$$
	
	La sensibilité permet de détecter le maximum de positif (plus elle est grande et plus on est sûr de détecter les cas positifs), tandis que la spécificité détecte le maximum de négatif. L'avantage de ces mesures est qu'elles ne dépendent pas de la proportion d'objets positifs ou négatifs.
	
	Ces mesures peuvent être portées sur une courbe ROC (Receiver operating characteristic). Si la sortie de l'algorithme d'apprentissage est un nombre (par exemple, la probabilité d'appartenir à une classe), on peut utiliser un seul afin de régler la sensibilité et la spécificité.
	
	\dessinS{74}{.5}
	
	Le meilleur algorithme est le D. Si un modèle retourne toujours la classe positive, il se trouvera dans le coin supérieur droit. S'il retourne toujours la classe négative, il sera dans le coin inférieur gauche. S'il renvoie une valeur au hasard, il sera situé au centre ; la diagonale représente les choix aléatoires biaisés.
	
	Si on se trouve dans la diagonale inférieure, on peut inverser les réponses ($+ \Rightarrow -$, $- \Rightarrow +$) pour revenir dans la diagonale supérieure, et donc avoir une meilleur AUC.
	
	\dessinS{75}{.5}
	
	On peut résumer une courbe ROC avec un nombre : la surface en dessous de la courbe. On peut l'interpréter comme la probabilité que deux objets choisis aléatoirement dans l'échantillon sont correctement ordonnés par le modèle, c'est-à-dire que le positif a un plus haut score que le négatif.
	
	On utilise d'autre mesures :
	
	\begin{itemize}
		\item la précision : $\frac{TP}{TP + FP}$, la proportion de bonnes prédictions parmi les prédictions positives
		\item le rappel (recall) : $\frac{TP}{TP + FN}$, la proportion de positifs qui sont détectés
		\item F-mesure = $\frac{2 * \text{précision} * \text{recall}}{\text{précision} + \text{recall}}$
	\end{itemize}
	
	\dessinS{76}{.5}
	
	Le meilleur algorithme aura une courbe ROC de la forme {\pigpenfont I}, tandis que la courbe de recall aura la forme {\pigpenfont C}.
	
	\subsection{Régression}
	
		A partir du moment où on a plus d'une classe, on utilise un taux d'erreur plutôt qu'une courbe ROC.
	
		\subsubsection{Erreur quadratique}
		
		$$\frac{1}{N} \sum_{i = 1}^N (y_i - \yh_i)^2$$
		
		Le carré permet de pénaliser les très mauvaises prédictions.
		
		\subsubsection{Erreur absolue moyenne}
		
		$$\frac{1}{N} \sum_{i = 1}^N \vert y_i - \yh_i \vert$$
		
		\subsubsection{Corrélation de Pearson}
		
		$$\frac{\sum_i (y_i - \frac{1}{N} \sum_j y_j)(\yh_i - \frac{1}{N} \sum_j \yh_j)}{(N - 1)s_y s_{\yh} }$$
		 		
		\subsubsection{Corrélation de rang de Spearman}
		
		$$1 - \frac{6 \sum_i d_i^2}{N(N^2 - 1)}$$
		
		avec $d_i$ la différence de rang de $y_i$ et $\yh_i$, le rang étant l'ordre obtenu après un tri des valeurs.
		
	\subsection{Mesures de performances pour l'entraînement}
	
	Les mesures de performances pour l'entraînement peuvent être différentes des mesures de performances de test. Il y a plusieurs raisons à cela :
	
	\begin{itemize}
		\item algorithmiquement, une mesure dérivable est soumise à une optimisation de gradient (par ex le taux d'erreur et l'erreur absolue moyenne ne sont pas dérivable, l'AUC n'est pas décomposable)
		\item le sur-apprentissage : pour l'entraînement, la perte incorpore souvent un terme de pénalité pour la complexité du modèle (ce qui est inutile au moment du test). De plus, certaines mesures sont moins promptes au sur-apprentissage (par exemple la marge).
	\end{itemize}
			
	\chapter{Machines à support vectoriel}

Cette méthode se base sur deux idées :

\begin{enumerate}
	\item la mise en place d'un classificateur à large marge, et
	\item le "noyautage" de l'espace d'entrée.
\end{enumerate}

\dessin{14}
Il faut trouver un classificateur linéaire. L'idée va être de trouver celui qui maximise la marge, c'est-à-dire la largeur du bord qui peut être étendue jusqu'à toucher une donnée.

\dessinS{15}{.45}

\section{Machines à support vectoriel linéaire}
	
	Soit un $\LS = \ens{(x_k, y_k)}^N_{k = 1}$, avec $y_k \in {-1, 1}$ et $x_k \in \mathbb{R}^n$. On cherche un classificateur de la forme
	
	$$\yh(x) = sgn(w^T x + b)$$
	
	qui classifie $\LS$ correctement, c'est-à-dire qui minimise
	
	$$\sum_{k = 1}^N1(y_k \neq \yh(x_k))$$
	
	\dessin{62}
	
	\subsection{Hyperplan de marge maximale}
	
	Lorsque les données sont linéairement séparables dans l'espace des features, l'hyperplan séparateur n'est pas unique.
	
	\dessin{55}
	
	Une SVM va chercher à maximiser la distance de l'hyperplan au point le plus proche dans $\LS$, autrement dit
	
	$$\underbrace{\max_{w, b} \underbrace{min \ens{\Vert x - x_k \Vert : w^T x + b = 0, k = 1, \dots , N}}_{\text{point le plus proche de la droite}}}_{\text{maximisation de la marge}}$$
	
	On maximise la marge car, intuitivement, c'est une méthode sûr. De plus, il existe des bornes théoriques sur l'erreur de généralisation qui dépend de la marge :
	
	$$Err(TS) < \bigoh (\frac{1}{\gamma})$$
	
	où $\gamma$ est la marge. Cependant, ces marges ne sont pas souvent atteintes. En pratique, une SVM fonctionne très bien.
	
	Cet algorithme conduit à un problème d'optimisation convexe, où la solution peut être écrite uniquement en terme de produits scalaires.
	
	\subsection{Problème d'optimisation}
	
	\dessin{56}
	
	$w$ est perpendiculaire à la ligne $y(x) = w^Tx + b$ :
	
	$$y(x_a) = 0 = y(x_b) \Rightarrow w^T(x_A - x_B) = 0$$
	
	Soit $x$ tel que $y(x) = 0$. La distance de l'origine à cette ligne est
	
	$$\Vert x \Vert \cos(w, x) = \Vert x \Vert \frac{w^T x}{\Vert w \Vert \: \Vert x \Vert} = \frac{w^Tx}{\Vert w \Vert} = \frac{-b}{\Vert w \Vert}$$
	
	Tout point $x$ peut être écrit comme
	
	$$x = x_{\perp} + r \frac{w}{\Vert w \Vert}$$
	
	où $x_{\perp}$ est la projection de $x$ et $\vert r \vert$ la distance entre $x$ et la ligne. En multipliant les deux membres par $w^T$, on obtient
	
	$$w^Tx = \underbrace{w^T x_{\perp}}_{\substack{-b\\ \text{car } x_{\perp} \text{ est sur} \\ \text{la droite}}} + r \frac{\overbrace{w^T w}^{\Vert w \Vert2}}{\Vert w \Vert}$$
	$$\Leftrightarrow r = \frac{w^Tx + b}{\Vert w \Vert} = \frac{y(x)}{\Vert w \Vert}$$
	
	Le problème d'optimisation peut alors être écrit comme 
	
	$$arg \max_{w, b} \ens{\frac{1}{\Vert w \Vert} \min_n [y_n . (w^T x_n + b)]}$$
	
	Si les exemples sont bien classés, $y_n$ et $w^Tx_n + b$ seront du même signe. $y_n$ permet de bien classer les valeurs, sinon on aurait une droite à l'infini (?).
	
	La solution n'est pas unique vu que l'hyperplan est inchangé si  on multiplie $w$ et $b$ par une constante $c > 0$ (par exemple, $c (w^Tx + b) = 0$ est aussi une solution). Pour imposer une unicité, on choisit typiquement $\vert w^T x + b \vert = 1$ pour le point $x$ qui est le plus proche de la surface (vecteur de support).
	
	Le problème est alors équivalent à maximiser $\frac{1}{\Vert w \Vert}$ (ou à minimiser $\Vert w \Vert$) avec les contraintes
	
	$$y_k(w^T x_k + b) \geq 1, \: \forall k = 1, \dots , N$$
	
	On a deux points qui sont les plus proches de la droite. C'est toujours le cas, sinon on pourrait encore augmenter la marge.
	
	Le problème de la SVM est équivalent à
	
	$$\min_{w, b} \varepsilon(w, b) = \frac{1}{2} \Vert w \Vert^2$$
	
	sujet aux $N$ contraintes
	
	$$y_k(w^T x_k + b) \geq 1, \: \forall k = 1, \dots , N$$

	$\Vert w \Vert$ devient $\frac{1}{2} \Vert w \Vert^2$ par facilité. Le $\frac{1}{2}$ permet de simplifier lors de la dérivation. Il s'agit d'un problème de programmation quadratique. Il existe une solution seulement si les données sont linéairement séparables, sinon des inéquations ne seront pas satisfaites.
	
	%Optimisation des contraintes : 10 $\rightarrow$ 17/45
	
	Soient $\alpha_k$, $k = 1, \dots N$ : on a le lagrangien qui doit être minimisé selon $w$ et $b$ et maximisé selon $\alpha$
	
	$$\mathcal{L}(w, b, \alpha) = \frac{1}{2} \Vert w \Vert - \sum_{k = 1}^N \alpha_k (y_k (w^Tx_k + b) - 1)$$
	
	Si on dérive par rapport à $w$ et $b$, on obtient les conditions optimales suivantes :
	
	$$\frac{\vartheta \mathcal{L}}{\vartheta w} = 0 \rightarrow w = \sum_{j = 1}^N \alpha_j y_j x_j$$
	$$\frac{\vartheta \mathcal{L}}{\vartheta b} = 0 \rightarrow \sum_{k = 1}^N \alpha_j y_j = 0$$
	
	Après substitution dans le lagrangien, on a
	
	$$\mathcal{L}(w, b, \alpha) = \frac{1}{2} \Vert w \Vert^2 - \sum_{k = 1}^N \alpha_k (y_k (w^Tx_k + b) - 1)$$
	
	Les conditions optimales conduisent au problème de maximisation dual :
	
	$$\max_{\alpha} \mathcal{W}(\alpha) = \sum_{k = 1}^N \alpha_k - \frac{1}{2} \sum_{i, j = 1}^N \alpha_i \alpha_j y_i y_j x_i^T x_j$$
	
	sujet aux $N$ inégalités
	
	$$\alpha_k \geq 0, \: \forall k = 1, \dots, N$$
	
	et à l'égalité
	
	$$\sum_{i = 1}^N \alpha_i y_i = 0$$
	
	
	\subsection{Vecteurs de support}
	
	Le problème primaire est donc
	
	$$\mathcal{L}(w, b, \alpha) = \frac{1}{2} \Vert w \Vert - \sum_{k = 1}^N \alpha_k (y_k (w^Tx_k + b) - 1)$$
	
	En accord avec les conditions complémentaires KKT (Karush-Kuhn-Tucker), le vecteur solution $w$ est tel que
	
	$$\alpha_k (y_k (w^T x_k + b) - 1) = 0, \: \forall k = 1, \dots , N$$
	
	$\alpha_k = 0$ si la contrainte est satisfaite comme une inégalité stricte $y_k (w^T x_k + b) > 1$, car c'est la façon de maximiser $\mathcal{L}$.
	
	$\alpha_k > 0$ si la contrainte est satisfaite comme une égalité $y_k(w^Tx_k + b) = 1$, auquel cas $x_k$ est le vecteur de support.
	
	Une fois que les valeurs optimales de $\alpha$ ont été déterminées, le modèle final peut être écrit comme
	
	$$\yh(x) = sgn(\sum_{i= 1}^N y_i \alpha_i x_i^T x + b)$$
	
	où les valeurs de $\alpha_k$ différentes de 0 (strictement positives) correspondent aux vecteurs de support. On a une sorte de KNN, mais on ne tient compte que des points sur la frontière, qui définissent les vecteurs de support.
	
	$b$ est calculé en exploitant le fait que pour tout $\alpha_k > 0$, on a nécessairement $y_k(wTx_k + b) - 1 = 0$.
	
	Borne pour le leave-one-out : 20/45
	
	Avoir un petit ensemble de vecteurs de support est efficaces, vu que seuls ces vecteurs $x_k$, leurs poids $\alpha_k$ et les labels de classe $y_k$ doivent être stockés pour classer de nouveaux exemples.
	
	Si un vecteur non-support $x'$ est retiré de l'échantillon d'apprentissage ou déplacé (en dehors de la région de la marge), la solution est inchangée.
	
	La proportion de vecteurs de support dans l'ensemble d'échantillon donne une borne sur l'erreur du leave-one-out :
	
	$$\text{Err}_\text{loo} \leq \frac{\vert \ens{k \vert \alpha_k > 0} \vert}{N}$$
	
	\subsection{Soft margin}
	
	A cause du bruit ou de données isolées (outliers), les données peuvent ne pas être linéairement séparables dans l'espace des features. 
	
	\dessin{57}
	
	Les discordances sont mesurées par les variables $\xi_i \geq 0$ avec la contrainte relaxée associée $y_i(w^Tx_i + b) \geq 1 - \xi_i$. En rendant $\xi_i$ suffisamment large, la contrainte peut toujours être satisfaite :
	
	\begin{itemize}
		\item si $0 < \xi_i < 1$, la marge n'est pas satisfaite mais $x_i$ est toujours correctement classé
		\item si $\xi_i > 1$, alors $x_i$ est mal classé.
	\end{itemize}
	
	\subsubsection{Marge douce en norme 1}
	
	Le problème primal est
	
	$$\min_{w, \xi} \frac{1}{2} \Vert w \Vert^2 + C \sum_{i = 1}^N \xi_i$$
	s.t.
	
	$$y_i(w^T x_i + b) \geq 1 - \xi_i, \: \xi_i \geq 0, \: \forall i = 1, \dots , N$$
	
	où $C$ est une constante positive et qui équilibre l'objectif de maximiser la marge et de minimiser l'erreur engendrée. Ainsi, si $C = 0$, on ne pénalise pas les $\xi_i$ donc les mauvais classements. Plus $C$ est grand et plus on augmente la complexité, car les $\alpha_k$ peuvent être grands et on veut un meilleur classement (donc on prend de plus en plus en compte les $\xi_i$).
	
	\dessin{77}
	
	On a le problème dual :
	
	$$\max_\alpha \mathcal{W}(\alpha) = \sum_{k = 1}^N \alpha_k - \frac{1}{2} \sum_{i, j = 1}^N \alpha_i \alpha_j y_i y_j x_i^T x_j$$
	
	s.t.
	
	$$0 \leq \alpha_k \leq C, \: \forall k = 1, \dots , N \text{ (contrainte de boîte)}$$
	$$\sum_{i = 1}^N \alpha_i y_i = 0$$
	
	\dessinS{58}{.5}
	
	Les points dans la marge correspondent à des $\xi > 0$. La figure du milieu illustre du sur-apprentissage.
	
\section{Kernel trick}

% Transparent passé : 24/45

Les données d'apprentissage peuvent ne pas être linéairement séparables dans l'espace d'entrée. On considère alors un mapping non linéaire $\phi$ vers un nouvel espace de features.

\dessin{59}

Le problème dual devient

$$\max_\alpha \mathcal{W}(\alpha) = \sum_{k = 1}^N \alpha_k - \frac{1}{2} \sum_{i, j = 1}^N \alpha_i \alpha_j y_i y_j \phi(x_i)^T \phi(x_j)$$

Plutôt que de définir le mapping $\phi$, on peut directement définir le produit scalaire $\phi^T(x) \phi(x')$, ce qui rend le mapping implicite.

Il est possible de caractériser mathématiquement les fonctions $K(x, x')$ définies sur des paires d'objets : cette fonction est appelée un noyau (positif ou de Mercer). Il peut cependant être difficile de trouver une mesure de la similarité entre $x$ et $x'$.

Le kernel trick est que n'importe quel algorithme qui utilise les données avec uniquement des produits vectoriels peut se baser sur ce mapping implicite, en remplaçant $x^Tx'$ par $K(x, x')$. Dans le cas du SVM, on a

$$\yh(x) = sgn(\sum_{k = 1}^N y_k \alpha_k \phi(x_k)^T\phi(x) + b) = sgn(\sum_{k = 1}^N y_k \alpha_k K(x, x_k) + b)$$

où les $\alpha_k$ peuvent être déterminés en résolvant le problème de maximisation quadratique 

$$\max_\alpha \mathcal{W}(\alpha) = \sum_{k = 1}^N \alpha_k - \frac{1}{2} \sum_{i, j = 1}^N \alpha_i \alpha_j y_i y_j K(x_i, x_j)$$

sujet aux $N$ contraintes d'inégalité

$$\alpha_k \geq 0, \: \forall k = 1, \dots , N$$

et la contrainte d'égalité

$$\sum_{i = 1}^N \alpha_i y_i = 0$$

\section{Notion mathématique du noyau}

Soit $U$ un ensemble d'objets non vide. Un noyau positif est une fonction $K(., .)$

$$K(.,.) : U \times U \rightarrow \mathbb{R}$$

telle que pour tout $N \in \mathbb{N}$ et pour tout $o_1, \dots , o_N \in U$, la matrice $N \times N$

$$K : K_{i, j} = K(o_i, o_j)$$

est symétrique et définie semi-positive.

Pour tout noyau positif $K$ défini sur $U$, il existe un espace de produit scalaire $\mathcal{V}$ et une fonction $\phi(.) : U \rightarrow \mathcal{V}$ tel que

$$K(o, o') = \phi(o) \times \phi(o')$$

où l'opérateur $\times$ dénote le produit scalaire dans $\mathcal{V}$.

En général, l'espace $\mathcal{V}$ n'est pas nécessairement de dimension finie.

Le noyau défini un produit scalaire, et donc une norme et une mesure de distance sur $U$, qui est héritée de $\mathcal{V}$ :

$$d^2_U(o, o') = d^2_\mathcal{V}(\phi(o), \phi(o')) = (\phi(o) - \phi(o'))^T(\phi(o) - \phi(o'))$$
$$= \phi(o) \times \phi(o) + \phi(o') \times \phi(o') - 2 \phi(o) \times \phi(o')$$
$$= K(o, o) + K(o', o') - 2 K(o, o')$$

\section{Exemple de noyaux}

\begin{itemize}
	\item noyau constant : $K(o, o') \models 1$
	\item noyau linéaire défini sur des attributs numériques : $K(o, o') = \mathbf{a}^T(o) \mathbf{a}(o') = \langle \mathbf{a}(o), \mathbf{a}(o')\rangle$ (produit scalaire)
	\item noyau polynomial : si $d$ est le degré maximum $K(o, o') = (\langle \mathbf{a}(o), \mathbf{a}(o') \rangle + 1)^d$
	\item noyau de Hamming pour des attributs discrets : $K(o, o') = \sum_{i = 1}^m \delta_{\mathbf{a}_i(o), \mathbf{a}_i(o')}$ (nombre de composantes communes aux deux vecteurs)
	\item noyau de texte, qui calcule le nombre de sous-chaînes communes dans $o$ et $o'$ (dimension infinie si la taille du texte n'est à priori pas bornée)
	\item combinaison de noyaux :
	
	\begin{itemize}
		\item la somme de plusieurs noyaux (positifs) est toujours un noyau (positif)
		\item le produit de plusieurs noyaux est aussi un noyau
		\item noyaux polynomiaux : $\sum_{i = 0}^n a_i(K(x, x'))^i$ si $\forall i : a_i \geq 0$
	\end{itemize}
	
	\item $K(x, x') = (x^Tx')^2$ (voir 31/45)
	\item noyau gaussien/à base radiale : si $\sigma$ est l'étalement de la distribution, $K(x, x') = \exp{\frac{-(x - x')^T(x - x')}{2 \sigma^2}}$
\end{itemize}

\dessinS{60}{.55}

\section{Méthodes à noyau}

\dessin{61}

L'approche est modulaire : on découple l'algorithme de la représentation. Beaucoup d'algorithmes peuvent utiliser des noyaux : ridge regression, PCA, k-means, etc. Des noyaux ont été défini pour plusieurs types de données : séries temporelles, images, graphes, séquences, etc.

Les intérêts principaux des noyaux est que l'on peut travailler efficacement dans des espaces de dimension très élevée (potentiellement infinie) dès qu'un produit scalaire est facile à calculer, et qu'on peut appliquer des algorithmes classiques sur des données en toute généralité, sans être nécessairement vectorielles (par exemple construire un modèle linéaire sur un graphe).

\section{Forces et faiblesses}

\begin{itemize}
	\item[+] les SVM sont motivées théoriquement
	\item[+] classifier des plus efficaces
	\item[+] implémentations efficaces pour des problèmes larges
	\item[-] modèles en boîte noire
	\item[-] le choix d'un bon noyau est difficile et critique pour atteindre une bonne précision
\end{itemize}

L'optimisation convexe est un outil très utile en apprentissage, car beaucoup de problèmes peuvent être formulés comme des problèmes d'optimisation convexe. On bénéficie également du travail important donné pour créer des algorithmes d'optimisation efficaces, et qui plus est donnent une solution unique (ce qui rend le problème plus stable).
	\chapter{Méthodes d'ensemble}

Pour une méthode donnée, il y a généralement un trade-off à faire entre le biais et la variance. Il est possible d'améliorer le modèle (par exemple avec du pruning pour les arbres de décision), mais ce n'est pas toujours possible. Le but des méthodes d'ensemble est de modifier de trade-off, quitte à perdre des features de la méthode initiale.


L'idée est de combiner plusieurs modèles construits avec un algorithme d'apprentissage afin d'améliorer la précision. Les arbres de décision sont souvent utilisés pour des raisons d'efficacité.

Il existe deux familles de méthode :

\begin{itemize}
	\item les techniques de moyennage, où la prédiction finale n'est que la moyenne des prédictions, et qui permettent de surtout faire diminuer la variance.
	\item les algorithmes de boosting, où on crée des modèles séquentiellement afin de diminuer le biais.
\end{itemize}

\section{Bagging}

	\subsection{Approche théorique}
	Supposons que l'on puisse générer plusieurs échantillons d'apprentissages $\ens{\LS_1, \LS_2, \dots , \LS_T}$ à partir de la distribution des données originales $P(\xu, y)$. Pour chacun des $\LS_i$, on va apprendre un modèle $\yh_{\LS_i}$ et on va calculer la moyenne :
	
	$$\yh_{\text{ens}} = \frac{1}{T} \sum_{i = 1}^T \yh_{\LS_i}(\xu)$$
	
	Pour rappel, on a l'erreur moyenne suivante :
	\begin{eqnarray*}
	\els{\text{Err}(\xu)} & = & \eyx{(y - h_B(\xu))^2} \\
	 & + & (h_B(\xu) - \els{\yh(\xu)})^2 \\
	 & + & \els{(\yh(\xu) - \els{\yh(\xu)})^2}
	\end{eqnarray*}
	
	Le biais n'est pas différent par rapport à l'algorithme original. En effet,
	
	$$\text{E}_{\LS_1, \dots , \LS_T} \ens{\yhens (\xu)} = \frac{1}{T} \sum_i \text{E}_{\LS_i}\ens{\yh_{\LS_i}(\xu)} = \els{\yh_{\LS}(\xu)}$$
	
	Par contre, la variance est divisée par $T$ :
	
	$$\text{E}_{\LS_1, \dots , \LS_T} \ens{(\yh_{\text{ens}}(\xu) - \text{E}_{\LS_1, \dots , \LS_T} \ens{\yh_{\text{ens}}(\xu)})^2} = \frac{1}{T} \els{(\yh(\xu) - \els{\yh_{\text{ens}}(\xu)})^2}$$
	
	En effet, différents échantillons d'apprentissage conduisent à différents modèles, surtout si l'algorithme sur-apprend les données. Vu qu'il n'y a qu'un seul modèle optimal, la variance est la source d'erreur.
	
	
	\subsection{En pratique}
	
	Il n'est généralement pas possible de créer plusieurs $\LS$, car $P(\xu, y)$ est inconnu, et les méthodes d'ensembles nécessitent justement beaucoup de données. L'idée est d'utiliser du bootstrap sampling pour générer plusieurs ensembles d'apprentissage.

	\dessin{21}
	
	On a alors l'algorithme du bagging (\textbf{b}ootstrap \textbf{agg}regat\textbf{ing}) :
	
	\begin{itemize}
		\item on crée $T$ bootstrap samples $\ens{B_1, \dots , B_T}$ à partir de $\LS$
		\item on apprend un modèle $\yh_{B_i}$ pour chaque $B_i$
		\item on construit le modèle de moyenne :
		
		$$\yh_{\text{ens}}(\xu) = \frac{1}{T} \sum_{i = 1}^T \yh_{B_i}(\xu)$$
	\end{itemize}
	
	\dessinS{90}{.45}
	
	Dans le cas d'une classification, $\yh(\xu)$ sera la classe majoritaire parmi $\ens{\yh_1(\xu), \dots , \yh_T(\xu)}$

	\dessinS{92}{.45}
	
	La variance est réduite, mais le biais augmente un peu car la taille effective du bootstrap sample est environ 30\% plus petite que le $\LS$ original ; tous les exemples de départ ne s'y trouvent pas, seulement environ $\sim$ 63.2\%.
	
	\dessinS{100}{.65}
	
	L'erreur est moindre, par contre il faut souvent augmenter la complexité pour obtenir l'erreur minimale.
	
	\subsection{Autres techniques de moyennage}
	
	Paradigme de perturbation et combinaison : on perturbe les données ou l'algorithme d'apprentissage pour obtenir plusieurs modèles qui sont bons sur l'échantillon d'apprentissage, puis on combine les prédictions des modèles.
	
	La variance est généralement diminuée (grâce à la moyenne), mais le biais augmente un peu à cause de la perturbation.
	
	Exemples : le bagging perturbe l'échantillon d'apprentissage, les réseaux de neurones peuvent être initialisés avec des poids aléatoires, random forests.
	
\section{Random Forests}

	C'est un algorithme de type \textit{perturb and combine} conçu spécifiquement pour les arbres. Elle utilise du bagging et une sélection aléatoire d'un ensemble d'attributs. L'algorithme suivant est utilisé :
	
	\begin{itemize}
		\item on construit l'arbre sur un bootstrap sample
		\item au lieu de choisir le meilleur split sur tous les attributs (au nombre de $p$), on sélectionne le meilleur split parmi un sous-ensemble de $k$ attributs
	\end{itemize}
	
	Il y a un trade-off biais/variance avec $k$ : plus $k$ est petit et plus la réduction est grande, mais plus haut sera le biais. Faire la moyenne casse le compromis biais/variance habituel.
	
	Le fait d'effectuer des choix aléatoires fait que l'arbre ne dépend pas de l'échantillon d'apprentissage.
	
	\dessinS{93}{.5}
	
	Un autre avantage des random forests est de diminuer le temps de calcul par rapport au bagging, car seul un sous-ensemble d'attributs est considéré lorsqu'on split un noeud (et pas tous). Généralement, $k = \sqrt{p}$.
	
	
\section{Décomposition de l'ambiguïté}

Supposons que l'on ait $T$ modèles $\ens{\yh_1, \dots , \yh_T}$ et leur moyenne

$$\yhens(\xu) = \frac{1}{T} \sum_i \yh_i(\xu)$$

\begin{eqnarray*}
\frac{1}{T} \sum_i \eyx{(y - \yh(\xu))^2} & = & \frac{1}{T} \sum_i \eyx{(y - \yhens(\xu) + \yhens(\xu) - \yh_i(\xu))^2} \\
& = & \frac{1}{T} \sum_i \eyx{(y - \yhens(\xu))^2} \\
& + & \frac{1}{T} \sum_i \underbrace{\eyx{(\yhens(\xu) - \yh_i(\xu))^2)}}_{\star_1} \\
& + & \underbrace{\frac{1}{T} \sum_i \eyx{\underbrace{(y(\xu) - \yhens(\xu))}_{\substack{\text{ne dépend} \\ \text{pas de }i}}
\underbrace{(\yhens(\xu) - \yh_i(\xu))}_{\substack{\text{ne dépend} \\
 \text{pas de } y}}}}_{\eyx{y - \yhens(\xu)}\frac{1}{T} \sum_i (\yens(\xu) - \yh_i(\xu)) \star_2} \\
\end{eqnarray*}
 
 $\star_1$ : on peut supprimer la somme car l'intérieur ne dépend pas de $y$.
 
 $\star_2$ : $\yens(\xu)$ ne dépend pas de $i$, donc on peut faire rentrer la somme et annuler le terme car $\sum_i \yh_i(\xu) = \yens(\xu)$.
 
 Au final, on a

\begin{eqnarray*}
& \frac{1}{T}\sum_i \eyx{(y - \yh_i(\xu))^2} & = \eyx{(y - \yhens(\xu))^2} + \frac{1}{T} \sum_i (y_i(\xu) - \yhens(\xu))^2 \\
\Leftrightarrow & \underbrace{\eyx{(y - \yhens(\xu))^2}}_{\substack{\text{Erreur de la} \\ \text{méthode d'ensemble}}} & = \underbrace{\frac{1}{T}\sum_i \eyx{(y - \yh_i(\xu))^2}}_{\substack{\text{Moyenne des} \\ \text{erreurs des modèles}}} - \underbrace{\frac{1}{T} \sum_i (y_i(\xu) - \yhens(\xu))^2}_{\substack{\text{Ambiguïté de la} \\ \text{méthode d'ensemble}}}
\end{eqnarray*}

L'ambiguïté mesure la variabilité des prédictions des modèles individuels. A moins que tous les modèles ne soient les mêmes, ce terme sera toujours positif. Du coup, plus les sous-modèles sont diversifiés et plus l'ambiguïté sera grande, donc plus la méthode d'ensemble sera efficace.

Le modèle moyen est donc toujours meilleur que les modèles individuels en moyenne. Ce n'est cependant pas vrai en classification.
	
\section{Boosting}
	
L'idée est de combiner plusieurs modèles "faibles", afin de produire un modèle plus puissant. Un modèle est considéré comme faible s'il a un grand biais (en classification, si le modèle est à peine meilleur qu'un choix aléatoire).

Les différences par rapport aux autres méthodes d'ensemble sont que

\begin{itemize}
	\item les modèles sont construits séquentiellement sur des versions modifiées des données
	\item les prédictions des modèles sont combinées à travers une somme pondérée/un vote
\end{itemize}

\dessinS{94}{.5}

Par exemple, pour passer de $\LS_1$ à $\LS_2$, on ignore les points bien classés de $\LS_1$.

En régression,

$$\yh(\xu) = \beta_1\yh_1(\xu) + \dots + \beta_T \yh_T(\xu)$$

En classification, $\yh(\xu)$ sera la classe majoritaire dans $\ens{\yh_1(\xu), \dots , \yh_T(\xu)}$, en tenant compte des points $\ens{\beta_1, \dots , \beta_T}$.

	\subsection{Adaboost}
	
	On suppose que l'algorithme d'apprentissage utilisé accepte les objets pondérés : $\ens{(x_1, y_1, w_1), \dots , (x_N, y_N, w_N)}$. C'est le cas de beaucoup d'algorithmes ; avec les arbres, on prend en compte les poids quand on compte les objets ; avec un réseau de neurones, on minimise l'erreur au carré pondérée.
	
	A chaque étape, Adaboost augmente les poids dans le cas où l'échantillon est mal classé par le modèle, ainsi l'algorithme se focalise sur les cas compliqués de l'échantillon d'apprentissage. $\beta_i$ sera ainsi plus petit si le modèle commet beaucoup d'erreurs. Par exemple,
	
	\begin{itemize}
		\item si l'exemple est mal classé, $w_i \leftarrow w_i \exp{\beta}$
		\item si l'exemple est bien classé, $w_i \leftarrow w_i . 1$
	\end{itemize}
	
	On a l'algorithme suivant : il prend en entrée un algorithme d'apprentissage et un échantillon d'apprentissage $\ens{(x_i, y_i) : i = 1, \dots , N}$. On initialise les poids $w_i = \frac{1}{N}$, $i = 1 , \dots , N$.
	
	Pour $t = 1$ jusque $T$ :
	
	\begin{itemize}
		\item on construit un modèle $\yh_t(x)$ avec l'algorithme d'apprentissage en utilisant les poids $w_i$
		\item on calcule l'erreur pondérée :
		
		$$\text{Err}_t = \frac{\sum_i w_i . I(y_i \neq \yh_t(x_i))}{\sum_i w_i}$$
		
		\item on calcule $\beta_t = \log{\frac{1 - \text{Err}_t}{\text{Err}_t}}$
		\item on met à jour les poids :
		
		$$w_i \leftarrow w_i . \exp{\beta_t . I(y_i \neq \yh_t(x_i))}$$
		
		\item on normalise les poids, de façon à ce que $\sum_i w_i = 1$
	\end{itemize}
	
	\dessinS{101}{.5}
		
	\subsection{Boosting aux moindres carrés}
	
	Cet algorithme de boosting est conçu pour les régressions. Il prend en entrée un algorithme d'apprentissage et un échantillon d'apprentissage $\ens{(x_i, y_i) : i = 1, \dots , N}$. On initialise de la façon suivante :
	
	$$\yh_0(x) = \frac{1}{N} \sum_i y_i$$
	$$r_i = y_i, \: i = 1, \dots , N$$
	
	Pour $t = 1$ jusque $T$ :
	
	\begin{itemize}
		\item pour $i = 1$ jusque $N$, on calcule les résiduels :
		
		$$r_i \leftarrow r_i - \yh_{t - 1}(x_i)$$
		
		\item on construit un arbre de régression sur l'échantillon d'apprentissage
		
		$$\ens{(x_i, r_i) : i = 1, \dots , N}$$
	\end{itemize}
	
	On retourne ensuite le modèle définit par
	
	$$\yh(x) = \yh_0(x) + \yh_1(x) + \dots + \yh_T(x)$$
	
	
	\subsection{Algorithme de boosting générique}
	
	Le but est de trouver $\yh(\xu) = \beta_1 \yh_1(\xu) + \dots + \beta_T \yh_T(\xu)$ et qui minimise $\sum_{i = 1}^N L(y_i, \yh(\xu_i))$.
	
	On utilise du \textit{forward stage-wise additive modeling} :
	
	\begin{enumerate}
		\item On initialise $\yh(\xu) = 0$
		\item pour $t = 1$ jusque $T$ :
		
		\begin{enumerate}
			\item on calcule
			
			$$(\beta_t, \yh_t) = arg \min_{\beta, \yh'} \sum_i L(y_i, \yh(\xu_i) + \beta \yh'(\xu_i))$$
			\item 
			
			$$\yh(\xu) \leftarrow \yh(\xu) + \beta_t \yh_t(\xu)$$
		\end{enumerate}
	\end{enumerate}
	
	On a ainsi
	
	\begin{itemize}
		\item pour le boosting aux moindres carrés, $L(y, y') = (y - y')^2$
		\item pour Adaboost, $L(y, y') = \exp{(-y y')}$
	\end{itemize}
	
	\dessinS{102}{.45}
	
	La courbe rouge (squared error) n'est pas intéressante, car elle pénalise les modèles où $y.f > 1$, or il faudrait qu'elle tende vers 0 comme les autres.
	
	\subsection{$\LS$ boosting}

	\remarque{Je n'ai pas retrouvé dans les transparents quoi que ce soit qui ait un rapport à cette section. Cela vient de mes notes, mais elles sont assez lacunaires, du coup ce qui suit ne veut pas dire grand chose.}
		
	On a un paramètre $u \in [0, 1]$ qui permet de ralentir l'évolution de l'algorithme, et donc le sur-apprentissage.
	
	\begin{center}
	\begin{tabular}{ccc|ccc}
	$x_1$ & \dots & $x_p$ & $y$ & $y_0$ & $y_1$ \\ 
	\hline 
	  &   &   & $y_1$ & $y_1 - u\yo$ & ? \\ 
	  &   &   & $y_2$ & $y_2 - u\yo$ & ? \\ 
	  &   &  & \vdots & \vdots & ? \\ 
	\end{tabular} 
	\end{center}
	
	Donc, à chaque itération $i$, on utilise ce qui a été obtenu à l'itération $i - 1$ avec un vecteur $u$, comme on ferait avec une moyenne exponentielle.
	
	\subsection{Autres méthodes de boosting}
	
	Il existe de nombreux autres algorithmes de boosting (par exemple le gradient boosting).
	
	Le boosting sur les arbres de décision/régression améliore très fortement leur précision, cependant on est beaucoup plus sensible au bruit (sur-apprentissage) que les techniques de moyennage.
	
	Pour que le boosting fonctionne, il faut que les modèles ne soient pas parfaits sur l'échantillon d'apprentissage. Pour les arbres, on peut soit les élaguer (pre-prune ou post-prune avec de la validation croisée), soit limiter le nombre de tests (et spliter d'abord sur les noeuds les plus impures). Il y a encore une fois un trade-off biais/variance qui dépend de la taille de l'arbre.
	
\section{Interprétabilité et efficacité}
	
Lorsque les méthodes d'ensemble sont combinées avec les arbres de décision, ils perdent de l'interprétabilité et de l'efficacité. En revanche, on les utilise toujours pour calculer l'importance des variables, en effectuant la moyenne sur tous les arbres. De plus, les méthodes d'ensemble peuvent être parallélisées et l'algorithme boosting utilise des petits arbres, ce qui fait que le coût en temps processeur n'est pas important.
		
	\dessinS{91}{.45}
	
\section{Autres approches d'ensemble}

	\subsection{Moyenne de modèles de Bayes}
	$$P(y \vert x, \LS) = \sum_{u \in \mathcal{H}} P(y \vert h, \LS) P(h \vert \LS)$$
	
	$$P(h \vert \LS) \propto P(h) P(\LS \vert h) \propto P(h) \sum_\theta P(\LS \vert \theta, h) P(\theta \vert h)$$
	
	\subsection{Stacking}
	
	On apprend un modèle qui combine des modèles. Soit $\LS = \ens{(x_i, y_i) : i = 1, \dots , N}$ et soit $A^T$ ($t = 0, \dots , T$) $T + 1$ algorithmes d'apprentissage.
	
	Pour $t = 1, \dots , T$ :
	
	\begin{itemize}
		\item construire un modèle :
		
		$$\yh^t = A^t(\LS)$$
		
		\item calculer les prédictions :
		
		$$\yh_i^t = \yh^t (x_i)$$
	\end{itemize}
	
	Soit $\LS^0 = \ens{(x_i^0, y_i)}$ avec $x_i^0 = (y_i^t)^T_{t = 1}$ : on retourne $\yh = A^0(\LS^0)$.
	
	Il s'agit d'une généralisation du boosting, dont le modèle est une combinaison linéaire des sous-modèles. L'algorithme d'apprentissage s'entraîne sur les sorties des sous-modèles.
	
	\dessin{103}
	
	Lors de la phase d'apprentissage des sous-modèles, il faut que les $y_i$ soient différents, par exemple avec de la validation croisée.
	
	\dessin{104}
	
\section{Conclusion}

Les méthodes d'ensemble sont très efficaces pour réduire le biais et/ou la variance, en transformant une méthode pas si bonne en une méthode très compétitive. Adaboost avec des arbres est considéré comme un des meilleurs algorithmes de classification.

L'interprétabilité et l'efficacité du modèle sont difficiles à préserver si on veut réduire la variance significativement.
	\chapter{Sélection de features}

Réduire le nombre de variable est utile car

\begin{itemize}
	\item on évite le sur-apprentissage (car on apprend sur des variables inutiles) et on améliore les performances du modèle
	\item on améliore l'interprétabilité
	\item les modèles sont plus rapides et moins coûteux
	\item on réduit tous les temps de calculs (si la sélection est elle-même rapide)
\end{itemize}

Il existe deux familles de méthodes :

\begin{itemize}
	\item la sélection de features : on cherche un petit (ou le plus petit) sous-ensemble de features qui maximisent la précision du modèle
	\item le ranking de features : on trie les variables selon leur importance à la prédiction de la sortie
\end{itemize}

A noter qu'on peut obtenir une sélection de feature avec un ranking : il suffit de sélectionner les $k$ premières features.

Trois approches existent dans la sélection de features :

\begin{itemize}
	\item filtre : sélection à-priori des variables, indépendamment d'un algorithme d'apprentissage supervisé
	\item embarqué (embedded) : sélection de features embarquée dans un algorithme d'apprentissage
	\item wrapper : utiliser la validation croisée pour trouver l'ensemble optimal de features pour un algorithme
\end{itemize}

\section{Techniques de filtre}

	L'idée est d'associer un score à chaque feature et de retirer celle dont le score est trop bas.
	
	Généralement, on utilise une fonction de score univariable, comme avec les arbres de décision ou avec des tests statistiques (t-test, chi$^2$, etc). Le nombre optimal de features peut être déterminé par de la validation croisée.
	
	Des approches de sélection multi-variable existent (dans les arbres de décision, etc), et peuvent être utile ; des features pourraient être inutiles toutes seules (on a un petit score univariable), mais ensembles, elles peuvent parfaitement expliquer la classification.
	
	\dessin{95}
	
	Avantages et inconvénients :
	\begin{itemize}
		\item[+] l'approche univariable est rapide et scalable
		\item[+] on est indépendant de l'algorithme d'apprentissage supervisé
		\item[-] on ignore l'algorithme d'apprentissage, et donc potentiellement une bonne combinaison de variables
		\item[-] l'approche univariable ignore la dépendance entre les features
		\item[-] l'approche multivariable est beaucoup plus lente
	\end{itemize}
	
\section{Techniques embarquées}

Certains algorithmes d'apprentissage supervisé embarquent une sélection de feature ; la recherche d'un sous-ensemble optimal de features se fait dans l'algorithme même.

Exemples :

\begin{itemize}
	\item le splitting des arbres de décision
	\item les méthodes ensemble à base d'arbre
	\item les valeurs absolues des points dans une SVM linéaire : $\yh(\xu) = sgn(\sum_i w_i x_i + b)$
\end{itemize}

Avantages et inconvénients :
	
\begin{itemize}
	\item[+] généralement efficace en terme de temps de calcul
	\item[+] bonne intégration avec l'algorithme d'apprentissage
	\item[+] multivariable
	\item[-] spécifique à l'algorithme d'apprentissage
\end{itemize}

	\subsection{LASSO}
	
	Il s'agit d'un modèle linéaire avec une pénalisation L1, comme de la ridge regression ; le carré est remplacé par une valeur absolue.
	
	$$\min_{\beta} \sum_{i = 1}^N (y_i -(\beta_0 + \sum_j \beta_j x_j))^2 + \lambda \sum_j \vert \beta_j \vert$$
	
	Les courbes rouges représentent la valeur des premiers termes, sans régularisation.
	
	\dessin{96}
	
	
	Le problème est identique à
	
	$$\min_{\beta} \sum_i (y_i - \beta \xu_i)^2$$
	$$\text{tel que } \sum_j \vert \beta_j \vert < C \text{ (losange)}$$
	$$\text{ou  } \sum_j  \beta_j^2  < C \text{ (cercle)}$$
	
	Un losange est mieux car on a plus de chance d'arriver au sommet d'un losange qu'au sommet d'un cercle, donc d'avoir un $\beta_i$ nul. Pour $\vert \beta \vert^p$, si $p \leq 1$, on se dirige vers une sorte d'étoile à quatre branches (\ding{71}), ce qui augmente encore plus la probabilité.
	
	Si $\lambda$ augmente, les coefficients rétrécissent ; on obtient un classement des variables (celle qui dure le plus longtemps est la plus importante).
	
	
	Cette méthode est plus lente que la ridge regression, car il y a un problème d'optimisation, et on est limité aux méthodes linéaires.
	
\section{Techniques de wrapper}

On essaie de trouver le sous-ensemble de features qui maximise la qualité du modèle induite par l'algorithme d'apprentissage. Cette qualité est estimée par de la validation croisée  vu que le nombre de sous-ensemble d'un ensemble de $p$ features est de $2^p$, tous les sous-ensembles ne sont pas évalués et des heuristiques sont nécessaires.

Plusieurs approches existent :

\begin{itemize}
	\item forward (ou backward) selection : on ajoute (retire) les variables qui diminue le plus (augmente le moins) l'erreur
	\item optimisation avec des algorithmes génétiques
\end{itemize}

Une approche populaire est l'élimination récursive de features. On suppose que l'on dispose d'un algorithme d'apprentissage qui peut ranker les features (ex : SVM, arbres de décision). A partir de l'ensemble complet de features, on itère :

\begin{itemize}
	\item on apprend un modèle à partir de l'ensemble courrant de features
	\item on calcule le rank des features avec le modèle
	\item on retire la feature avec le plus petit rank
\end{itemize}

Au final, on garde l'ensemble de features qui donne la plus petite erreur par validation croisée.

\dessinS{97}{.45}

Avantage et inconvénients :

\begin{itemize}
	\item[+] l'ensemble de features est taillé sur mesure pour l'algorithme d'apprentissage
	\item[+] il est possible de trouver des interactions et de supprimer des variables redondantes
	\item[-] méthode susceptible au sur-apprentissage : il est parfois facile de trouver un petit sous-ensemble de variables bruitées qui donne de très bons résultats
	\item[-] coûteux, car il faut construite un modèle sur chaque sous-ensemble de variable
\end{itemize}


\section{Biais de sélection}

\paragraph{Mauvaise méthode de sélection}

A partir de la base de données, on sélectionne les $N$ meilleurs variables avec un filtre. On évalue ensuite un algorithme qui utilise ces $N$ variables par validation croisée sur la base de données.

Supposons une expérience artificielle où les variables sont choisies aléatoirement, de même que la classe de sortie. On effectue alors deux essais :

\begin{itemize}
	\item tree bagging sans sélection de feature : erreur en validation croisée 10-fold = 52\%
	\item idem mais avec les 20 meilleures features : 35\%
\end{itemize}

On pourrait supposer qu'il y a 20 variables intéressantes et qu'avec elles on pourrait créer un meilleur classificateur qu'un classificateur aléatoire. Cependant, sur un nouvel ensemble d'échantillons, on obtient une erreur de 52\%. De plus, le 35\% obtenu n'est pas possible, car toutes les valeurs sont aléatoires : tout modèle aura une erreur de 50\%.

On a un problème de sur-apprentissage, car on a sélectionné et testé le modèle sur base de toute la base de données. Le bon protocole aurait dû être

\begin{itemize}
	\item diviser $\LS$ en 10 sous-ensembles
	\item pour $i = 1$ jusqu'à 10 :
	
	\begin{itemize}
		\item retirer le $i$ème sous-ensemble de $\LS$
		\item sélectionner les 20 variables sur les sous-ensembles restant
		\item apprendre le modèle sur les 20 variables et les objets restant
		\item tester le modèle sur le $i$ème sous-ensemble de variables
	\end{itemize}
\end{itemize}

Il faut donc utiliser une table différente pour la sélection et la validation, afin d'éviter de la corrélation lors de la validation.
	
	\part{Apprentissage non supervisé}
	\chapter{Apprentissage non supervisé}

Le but de l'apprentissage non supervisé est de trouver des irrégularités dans les données, sans se soucier de la relation entrée-sortie. On recherche ainsi les groupes de variables ou d'objets intéressants, et des dépendances entre les variables.

Il existe trois grandes familles de problèmes :

\begin{itemize}
	\item clustering : trouver des groupes d'échantillons ou de variables.
	\item réduction de dimensionnalité : on projette les données d'un espace à haute dimension vers un espace plus petit.
	\item estimation de densité : déterminer la distribution des données dans l'espace d'entrée.
\end{itemize}

\section{Clustering}

Le but est de grouper une collection d'objets en sous-ensembles (appelés clusters), de façon à ce que chaque objet dans un cluster soit proche des autres, tout en étant éloigné des objets des autres clusters.
	
Ces groupements peuvent être

\begin{itemize}
	\item des groupements de lignes/d'objets similaires
	\item des groupements de colonnes/variables
	\item du bi-clustering, c'est-à-dire en se basant sur les lignes et les colonnes.
\end{itemize}

\dessin{43}

Applications :

\begin{itemize}
	\item marketing : trouver des groupes de clients qui ont un comportement similaire, en se basant sur leurs caractéristiques et les achats précédents
	\item biologie : classifier de la faune et la flore selon leurs caractéristiques
	\item web : classification de documents (par exemple des articles de blog)
\end{itemize}

Deux composantes sont considérées :

\begin{itemize}
	\item la mesure de distance entre deux objets
	\item un algorithme de clustering, qui va minimiser les distances entre les objets d'un groupe et/ou maximiser les distances entre des groupes
\end{itemize}

	\subsection{Mesure de distances}
		\subsubsection{Distance Euclidienne}
		Elle mesure la différence entre des coordonnées et pénalise les grosses différences. Il s'agit de la racine carrée de la somme des carrés des différences entre les coordonnées :
		
		$$d_e(x_1, x_2) = \sqrt{(x_{10}-x_{20})^2 + (x_{11}-x_{21})^2 + \dots}$$
		
		\subsubsection{Distance de Manhattan}
		Elle mesure la différence entre des coordonnées, mais de manière robuste. Il s'agit de la somme des différences absolues de toutes les coordonnées :
		
		$$d_e(x_1, x_2) = \vert x_{10}-x_{20} \vert + \vert x_{11}-x_{21} \vert + \dots$$
		
		\subsubsection{Corrélation}
		Elle mesure une différence en tenant compte des tendances. La distance entre deux vecteurs est $1 - \rho$, où $\rho$ est la corrélation de Pearson entre les deux vecteurs :
		
		$$\rho(x_1, x_2) = \frac{cov(x_1, x_2)}{\sigma_{x_1} \sigma_{x_2}} = \frac{\sum_{i = 1}^n (x_{1, i} - \overline{x}_1)(x_{2, i} - \overline{x}_2)}{\sqrt{\sum_{i = 1}^n (x_{1, i} - \overline{x}_1)^2} \sqrt{\sum_{i = 1}^n (x_{2, i} - \overline{x}_2)^2}}$$
	
		On a que $\rho \in [-1, 1]$, donc $1 - \rho \in [0, 2]$ : 0 signifie que les données sont fortement corrélées.
	
	\subsection{Clustering hiérarchique}
	
		\subsubsection{Algorithme}
		On a l'algorithme suivant :
	
		\begin{enumerate}
			\item Chaque objet est assigné à son propre cluster
			\item Itérativement :
		
			\begin{itemize}
				\item les deux clusters les plus similaires sont joins et rassemblés en un.
				\item la matrice de distances est mise à jour avec le nouveau cluster qui en remplace deux.
			\end{itemize}
		\end{enumerate}
		
		\subsubsection{Distance entre deux clusters}
		
		On a plusieurs possibilités :
		
		\begin{itemize}
			\item Single linkage : utiliser la plus petite distance entre deux objets du cluster. Cela a tendance à créer des clusters étalés.
			
			\item Complete linkage : utiliser la plus grande distance entre deux objets du cluster. Cela a tendance à créer des grappes.
			
			\item Average distance : calculer la distance moyenne. On obtient un mix des deux autres mesures ; on a une sorte de distance entre les centres de masse.
		\end{itemize}
				
		\dessin{44}
		
		\dessin{45}
		
		\subsubsection{Dendrogramme}
		
		Cela permet de visualiser le clustering hiérarchique et de déterminer visuellement le nombre de clusters.
		
		\dessin{46}
		
		
		\subsubsection{Forces et faiblesses}
		
		\begin{itemize}
			\item[+] on n'a pas besoin de supposer un nombre particulier de cluster
			\item[+] on peut utiliser n'importe quel type de matrice de distance
			\item[+] on a parfois une interprétation facile des résultats
			
			\item[-] trouver une interprétation n'est pas toujours aisé
			\item[-] une fois qu'il a été décidé de combiner deux clusters, on ne peut pas revenir en arrière. Par exemple, le cluster rouge montre un mauvais départ, alors qu'on aurait voulu obtenir les deux clusters verts :
			
			\dessin{47}
			\item[-] pas très bien motivé théoriquement
		\end{itemize}
		
		\subsubsection{Algorithme de clustering combinatoire}
		
		Soit un nombre de clusters $K < N$ et un encodeur $C$ qui assigne la $i$ème observation au cluster $C(i)$. On va chercher la fonction $C^*$ qui minimise une fonction de perte, qui mesure si l'objectif de clustering est atteint.
		
		Par exemple, on pourrait avoir comme fonction de perte une qui se base sur l'éparpillement des objets d'un cluster (within cluster scatter) :
		
		$$W(C) = \frac{1}{2} \sum_{k = 1}^K \sum_{C(i) = k} \sum_{C(i') = k} d(x_i, x_{i'})$$
		
		Le nombre de possibilité est cependant trop grand pour une énumération.
		
		$$S(N, K) = \frac{1}{K!} \sum_{k = 1}^K(-1)^{K - k} \begin{pmatrix}
		K \\ 
		k
		\end{pmatrix}  k^N$$
	
	\subsection{K-means}
	
	Cet algorithme effectue un partitionnement avec un nombre $k$ fixé de clusters. On utilise la distance euclidienne entre deux objets, et on va chercher à minimiser la somme des variances intra-cluster :
	
	$$W(C) = \frac{1}{2} \sum_{k = 1}^K \sum_{C(i) = k} \sum_{C(i') = k} \Vert x_i - x_{i'} \Vert^2 = \sum_{k = 1}^K N_k \sum_{C(i) = k} \Vert x_i - \overline{x}_k \Vert^2$$
	
	avec $\overline{x}_k = (\overline{x}_{1k}, \dots, \overline{x}_{pk})$ le centre du cluster $k$ et $N_k$ le nombre de points dans le cluster $k$ :
	
	$$N_k = \sum_{i = 1}^N I(C(i) = k)$$
	
	Cela revient donc à un problème d'optimisation :
	
	$$min_{C, \ens{m_k}^K_1} \sum_{k = 1}^K N_k \sum_{C(i) = k} \Vert x_i - m_k Vert^2$$
	
		\subsubsection{Algorithme}
		
		\begin{enumerate}
			\item On assigne aléatoirement chaque point à un cluster
			\item Itérativement :
			
			\begin{itemize}
				\item calculer les moyennes des clusters $\ens{m_1 , \dots , m_K}$
				\item pour ces moyennes, assigner chaque observation à la moyenne de cluster la plus proche :
				
				$$C(i) = argmin_{1 \leq k \leq K} \Vert x_i - m_k \Vert^2$$
			\end{itemize}
		\end{enumerate}
		
		On s'arrête lorsqu'il n'y a plus de changement.
		
		\dessin{48}
		
		\dessin{49}
		
		\dessin{50}
		
		Chaque étape réduit l'éparpillement dans les clusters, donc la convergence est assurée, mais uniquement vers un optimum local.
		
		De plus, vu le départ aléatoire de l'algorithme, on pourrait avoir plusieurs solutions différentes. La solution est de redémarrer l'algorithme plusieurs fois.
		
		\subsubsection{K-medoids}
		
		Il s'agit d'une extension des k-means qui permet d'utiliser n'importe quelle mesure de distance. Elle est par contre beaucoup plus lente.
		
		\dessin{51}
		
		\subsubsection{Forces et faiblesses}
		
		\begin{itemize}
			\item[+] simple et facile à comprendre
			\item[+] on peut clusteriser n'importe quel nouveau point (contrairement au clustering hiérarchique)
			\item[+] bonne motivation théorique
			\item[-] il faut fixer le nombre de clusters
			\item[-] sensible au choix initial des centres des clusters
			\item[-] sensible aux données isolées (outliers)
		\end{itemize}
	
	\subsection{Self-Organizing Maps}
	
	C'est une méthode similaire au k-means, mais avec des contraintes supplémentaires : les clusters sont donnés sous forme de matrice (à une ou deux dimensions).
	
	\dessin{52}
	
		\subsubsection{Algorithme}
		
		Itérativement :
		\begin{itemize}
			\item prendre $P$ données aléatoirement		
			\item déplacer tous les noeuds dans la direction de $P$ : plus un noeud est dans la topologie mieux c'est (et inversement)
			\item diminuer la quantité de mouvement autorisée
		\end{itemize}
		
		
	\subsection{Nombre de cluster}
	
	La question est de savoir quand s'arrêter dans le clustering hiérarchique et comment choisir $k$ pour les k-means et les SOMs. On a un phénomène d'overfitting, comme en apprentissage supervisé :
	
	\begin{itemize}
		\item on sur-apprend les données s'il y a trop de clusters. On arrive ainsi à trouvers des clusters qui n'existent pas dans les données à cause du bruit
		\item on sous-apprend les données s'il y a trop peu de clusters, on passe ainsi à côté de clusters pertinents
	\end{itemize}
	
	Il n'est cependant pas possible, contrairement à l'apprentissage supervisé, d'effectuer une validation croisée.
	
	Un choix de nombre de cluster consiste à prendre le point d'inflexion de la courbe de variance intra-cluster.
	
	\dessin{53}
	
	
	Autres possibilités :
	
	\begin{itemize}
		\item utiliser des indices internes : sélectionner $k$ qui minimise/maximise les distances intra- et extra-cluster
		\item gap statistic : utiliser une méthode qui compare un indice interne avec ce qu'on aurait obtenu avec des données aléatoire, et choisir le $k$ qui minimise cette différence
		\item stabilité : sélectionner le $k$ qui conduit à des clusters stables (calculés avec une analyse bootstrap)
	\end{itemize}
	
	

\section{Réduction de dimensionnalité}

\section{Estimation de densité}
\end{document}	