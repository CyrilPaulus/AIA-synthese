\chapter{Introduction}
L'apprentissage supervisé consiste, à partir de la base de donnée (learning sample, échantillon de test), à trouver une fonction $f$ qui prend entrée les variables du problème et qui approxime au mieux la sortie :

$$\hat{Y} = f(X_1, X_2, X_3, X_4)$$

Plus formellement, l'apprentissage consiste, à partir d'un échantillon d'apprentissage $\ens{(x_i, y_i) \vert i = 1, \dots , N}$, avec $x_i \in \mathcal{X}$ et $y_i \in \mathcal{Y}$, à trouver une fonction $f : \mathcal{X} \rightarrow \mathcal{Y}$ qui minimise la fonction de probabilité de perte $l : \mathcal{Y} \times \mathcal{Y} \rightarrow \mathbb{R}$ sur la distribution jointe des paires d'entrées sorties : $E_{x, y} \ens{l(f(x), y)}$.

Cette fonction de perte $l$ prend en entrée deux sorties $Y$ et retourne 1 si elles sont équivalentes, 0 sinon.

Lorsque la sortie est une valeur symbolique, on parle de classification. Si la sortie est une valeur numérique, on parle de régression.

Un modèle sera déterministe s'il est parfait, c'est-à-dire s'il a une règle de classification qui ne commet par d'erreur.

\section{Sélection de modèle}

Un algorithme d'apprentissage est défini par

\begin{itemize}
	\item une famille de modèles candidats (un espace d'hypothèses $H$),
	\item une mesure de la qualité d'un modèle, et
	\item une stratégie d'optimisation.
\end{itemize}

L'algorithme va ainsi, à partir de l'échantillon d'apprentissage, retourner la fonction $h$ de $H$ de meilleur qualité.


\section{Comparaison des méthodes}
	
\dessin{23}

A noter que l'importance relative des critères dépend de l'application, et que ce ne sont que des tendances générales.
