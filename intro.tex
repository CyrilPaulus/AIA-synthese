\chapter{Introduction}

L'apprentissage consiste à

\begin{itemize}
	\item améliorer les performances d'un ordinateur dans certaines tâches, avec de l'expérience ;
	\item extraire un modèle d'un système en se basant sur les observations de ce systèmes dans certaines situations ;
	\item créer un modèle, c'est-à-dire une relation entre les variables utilisées pour décrire le système.
\end{itemize}

Les deux buts principaux de l'apprentissage sont la prédiction et la meilleure compréhension d'un système.

L'apprentissage est utilise quand il n'y a pas d'expertise humaine, quand les humains ne sont pas capables d'expliquer leur expertise, quand les solutions changent au cours du temps ou quand les solutions nécessitent d'être adaptées à des cas particuliers.

\dessinS{1}{.4}

L'exploration de données se déroulent en plusieurs étapes :

\begin{enumerate}
	\item Génération de données ;
	\item Préprocessing : normalisation des valeurs, traitement des valeurs manquantes, sélections d'une composante, etc ;
	\item Apprentissage : développement d'une hypothèse, choix d'un algorithme d'apprentissage, etc ;
	\item Validation d'hypothèse : validation croisée, déploiement du modèle, etc.
\end{enumerate}

	\section{Glossaire}
	
	\dessinS{2}{.4}

	\section{Protocoles d'apprentissage automatique}
	
	% TODO Apprentissage supervisé, non-supervisé, apprentissage par renforcement, apprentissage en mode batch vs apprentissage on-line, apprentissage semi-supervisé et transductif.  Donner un exemple de problème pratique et de méthode pour chaque type de protocole.