\subsection{$k$-NN - méthode des $k$ème plus proche voisin}
		
Cette méthode consiste à prédire la sortie en se basant sur les plus proches voisins de l'entrée.
		
\dessin{11}
		
Pour ce faire, on trouve les $k$ plus proches voisins, en utilisant la distance euclidienne. La sortie sera,
		
\begin{itemize}
	\item dans le cadre d'une classification, la classe la plus fréquente,
	\item dans le cadre d'une régression, la valeur moyenne.
\end{itemize}
		
\dessin{12}
		
\begin{itemize}
	\item[+] très simple ;
	\item[+] peut être adapté pour tout type de données, en changeant la mesure de la distance ;
	\item[-] choisir une bonne mesure de la distance est un problème compliqué ;
	\item[-] cet algorithme est très sensible à la présence de bruit ;
	\item[-] lent.
\end{itemize}