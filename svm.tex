\subsection{Machine à support vectoriel}

Cette méthode se base sur deux idées :

\begin{enumerate}
	\item la mise ne place d'un classifieur à large marge, et
	\item le "noyautage" de l'espace d'entrée.
\end{enumerate}

\dessin{14}
Il faut trouver un classifieur linéaire. L'idée va être de trouver celui qui maximise la marge, c'est-à-dire la largeur du bord qui peut être étendue jusqu'à toucher une donnée.

\dessin{15}

C'est une méthode intuitivement sûre, avec une borne théorique sur l'erreur : $E(TS) < \mathcal{O}(\frac{1}{\text{margin}})$.

65 $\rightarrow$ 74
